Die Aorta im menschlichen Körper hat einen Radius von
$r_{\text{Aorta}}=1.2\text{cm}$, in ihr fliesst das Blut mit einer
Geschwidigkeit von $v_{\text{Aorta}}=0.4\text{m/s}$.
Die Kapilaren dagegen haben $r_{\text{Kapillare}}=4\mu\text{m}$
und das Blut fliesst darin
mit einer Geschwindigkeit von $v_{\text{Kapillare}}=0.5\text{mm/s}$.
Wieviele Kapillaren gibt es?

\begin{loesung}
Die Kontinuitätsgleichung besagt, dass das pro Zeiteinheit durch
die Aorta strömende Blutvolumen gleich dem pro Zeiteinheit durch
die Kapillaren fliessende Volumen ist.
Es gilt also
\begin{align*}
\pi r_{\text{Aorta}}^2 v_{\text{Aorta}} 
&=
n\pi r_{\text{Kapillare}}^2 v_{\text{Kapillare}}
\\
n
&=
\frac{r_{\text{Aorta}}^2 v_{\text{Aorta}}}%
{r_{\text{Kapillare}}^2 v_{\text{Kapillare}}},
\end{align*}
wobei $n$ die Anzahl der Kapillaren ist.
Setzt man die Zahlenwerte ein erhält man
\begin{align*}
n
&=
\frac{
1.2^2\text{cm}^2\cdot 40\text{cm/s}
}{
(4\cdot 10^{-4})^2\text{cm}^2\cdot 5\cdot 10^{-2}\text{cm/s}
}
=
\frac{1.44\cdot 40}{16\cdot 10^{-8}\cdot 5\cdot 10^{-2}}
=
\frac{57.6}{8\cdot 10^{-9}}
=
7.2\cdot 10^{9}.
\end{align*}
Es gibt also etwa 7.2 Milliarden Kapillaren.
\end{loesung}

