%
% multipol.tex
%
% (c) 2017 Prof Dr Andreas Müller, Hochschule Rapperswil
%
\chapter{Multipolentwicklung, Kugelfunktionen und der
kosmische Mikrowellenhintergrund
\label{skript:chapter:multipol}}
\lhead{Multipole und CMB}
\rhead{}
Beobachtet man ein kleines Objekt aus einiger Distanz sind viele
Details nicht mehr erkennbar.
Dieses Ph"anomen muss auch einen mathematischen Ausdruck haben.

Eine Masseverteilung wird ein Gravitationsfeld haben, welches aus
grosser Distanz kaum unterscheidbar sein wird vom Gravitationsfeld
eines einzelnen Masspunktes.
Eine Ladungsverteilung erzeugt ein elektrisches Feld, welches in
grosser Entfernung mit $\frac1r$ abfällt.
Ist die Gesamtladung $0$, bedeutet das aber nicht, dass das
Feld vollständig verschwindet, es bleibt ein Restfeld, welches
allerdings mit $\frac1{r^2}$ abfällt.
In all diesen Fällen haben wir es mit einer Funktion $f(x,y,z)$ zu tun,
die für vergleichsweise kleine Werte der Koordinaten $x$, $y$ und $z$
kaum verstehen können.
Uns interessiert aber vor allem das Verhalten für grosse Werte der
Koordinaten, also weit von den unverständlichen Details entfernt.

Neben dem Abfall der Funktionswerte mit der Entfernung ist auch
die Verteilung der Funktionswerte in verschiedenen Richtungen wesentlich.
Wir erwarten also, dass wir jede beliebige Funktion in Terme der Form
$
f(r) g(\vartheta,\varphi)
$
zu zerlegen, worin $f(r)$ nur von der Entfernung und $g(\vartheta,\varphi)$
nur von geographischer Länge und Breite der Richtung abhängt.
Dabei ist immer noch möglich, das verschieden schnell abfallende
Terme verschiedene Richtungsverteilungen haben.

Wir werden in diesem Kapitel daher eine Familie von Funktionen
von $\vartheta$ und $\varphi$ entwickeln, die Kugelfunktionen,
mit welchen wir die Richtungsabhängigkeit untersuchen können.
Zusammen mit verschiedenen Funktionen, die die Entfernungsabhängigkeit
beschreiben, erhalten wir so ein Instrumentarium zur Analyse von
beliebigen Funktionen.
Diese Funktionen haben eine grosse Zahl von Anwendungen:

\begin{enumerate}
\item
Die Multipolentwicklung erlaubt elektromagnetische Felder in grosser
Entfernung von der Quelle zu beschreiben.
Man spricht auch vom Nah- und Fernfeld.
\item
Die Wahrscheinlichkeitsverteilung eines Elektrons im Feld einer Punktladung
fällt mit grosser Entfernung exponentiell schnell ab.
Der Abfall ist aber trotzdem nicht für alle Zustände gleich schnell,
die Unterschiede haben mit dem Drehimpuls des Elektrons um den Atomkern
zu tun, und wirken sich in möglichen Strahlungsmustern aus.
\item
Jede Funktion auf einer Kugeloberfläche kann zerlegt werden in eine
Summe von Funktionen ganz ähnlich wie eine periodische Funktion
zerlegt werden kann in eine Summe von Sinus- und Kosinus-Funktionen.
Dies wird zum Beispiel für die Analyse des Gravitationsfeldes von
Erde und Mond verwendet, oder für die Abweichung der Erdoberfläche
von der Form eines Rotationsellipsoids.
\item 
Die Beträge der Fourierkoeffizienten eines TODO Aussagen machen
über das Vorhandensein von geometrischen Features, während die exakte
Position dieser Features durch die Phase bestimmt ist.
Für die Kugelfunktionen gilt etwas ähnliches, daher eignen sie
sich dazu, geometrische Features in Funktionen auf der Kugeloberfläche
zu suchen, deren exakte Position nicht bekannt sind.
\item
Der kosmische Mikrowellenhintergrund kann mit Kugelfunktionen analysiert
werden.
Diese Analyse zeigt, dass das Universum flach ist, und damit dass der
gesamte Energieinhalt des Universums $0$ ist.
\end{enumerate}

%
% m-1dim.tex -- ein eindimensionales Beispiel
%
% (c) 2017 Prof Dr Andreas Müller, Hochschule Rapperswil
%
\section{Ein eindimensionales Beispiel%
\label{skript:multipol:1dimbeispiel}}
\rhead{Ein eindimensionales Beispiel}
In diesem Abschnitt betrachten wir eine Funktion, die nur von einer
Variablen $x$ abhängt.
Ausserdem interessiert uns in erster Linie das Verhalten der Funktion
für $x\to\pm\infty$.
Wir verwenden dazu eine Methode, die sich leicht auf die dreidimensionale
Situation ausweiten lässt.

\subsection{Monopol und Dipol}
Das Potential einer Ladung $e$ fällt mit der Entfernung $r$ nach dem
Gesetz
\[
f(r)=\frac1{4\pi\varepsilon_0}\frac{q}r
\]
ab.
Platziert man die Ladung $q$ stattdessen im Punkt $a$, dann ist das
Potential in Abhängigkeit von $x$
\begin{equation}
f_1(x) = \frac{q}{4\pi\varepsilon_0} \frac1{|x-a|}.
\label{skript:multipol:1monopol}
\end{equation}
Für $x>$ kann man \eqref{skript:multipol:1monopol} schreiben als
\begin{equation}
f_1(x)
=
\frac{q}{4\pi\varepsilon_0} \frac1{x-a}
=
\frac{q}{4\pi\varepsilon_0} 
\frac{\displaystyle\frac{1}{x}}{1-\displaystyle\frac{a}{x}}.
\label{skript:multipol:1abfall}
\end{equation}
Im letzte Term kann man ablesen, dass der Terme $a/x$ im Nenner
auf der rechten Seite für grosse $x$ unbedeutend wird, so dass $f(x)$
immer noch wie $1/x$ abfällt, unabhängig davon, wo die Ladung platziert
wird.

Zwei entgegengesetzte Punktladungen an den Stellen $x=\pm a$ haben also
das Potential
\begin{equation}
f_2(x)
=
\frac1{4\pi\varepsilon_0}\frac{q}{|x+a|}
-
\frac1{4\pi\varepsilon_0}\frac{q}{|x-a|}
=
\frac{q}{4\pi\varepsilon_0}\biggl( \frac1{|x+a|} - \frac1{|x-a|} \biggr).
\label{skript:multipol:1dipol}
\end{equation}
Wir sind nur an den Funktionswerten interessiert für $x$-Werte, die
wesentlich grösser sind als $a$, als in grosser Entfernung von
den Ladungen.
Wir interessieren uns also für $x$-Werte so, dass $x/a$ sehr gross ist,
oder umgekehrt dass $a/x$ sehr klein ist.

Wenn man in Gleichung \eqref{skript:multipol:1dipol} das Vorzeichen
von $x$ kehrt, dann ändert das Vorzeichen von $f$, also $f_2(-x)=-f_2(x)$,
$f$ ist eine ungerade Funktion.
Es reicht daher, die Funktion $f_2(x)$ für $x\gg a$ zu untersuchen.

Verwenden wir die Gleichung \eqref{skript:multipol:1abfall} in $f_2(x)$,
erhalten wir
\begin{align}
f_2(x)
&=
\frac{q}{4\pi\varepsilon_0}\biggl(\frac1{x+a}-\frac1{x-a}\biggr)
=
\frac{q}{4\pi\varepsilon_0}\cdot\frac1x\cdot\biggl(
\frac1{1+\frac{a}{x}}
-
\frac1{1-\frac{a}{x}}
\biggr)
=
\frac{q}{4\pi\varepsilon_0}\cdot\frac1x\cdot
\frac{\bigl(1-\frac{a}{x}\bigr)-\bigl(1+\frac{a}{x}\bigr)}{1-\frac{a^2}{x^2}}
\notag
\\
&=
\frac{q}{4\pi\varepsilon_0}\cdot\frac1x\cdot
\biggl(-\frac{2a}{x}\biggr)
\cdot
\frac1{1-\frac{a^2}{x^2}}
=
-\frac{qa}{2\pi\varepsilon_0}\cdot\frac1{x^2}\cdot\frac{1}{1-\frac{a^2}{x^2}}
\label{skript:multipol:2abfall}
\end{align}
Der Term $a^2/x^2$ im Nenner des letzten Faktors ist für grosse Werte von
$x$ wieder unbedeutend.
Der dominierende Term für das Verhalten für grosse $x$ ist daher der
Faktor $1/x^2$.

Ein weiterer interessanter Aspekt der Formel~\eqref{skript:multipol:2abfall}
ist, dass nur noch die Kombination $qa$ von Ladung und Abstand der Ladungen
vorkommt.
Das Fernfeld für grosse Werte von $x$ ändert also nicht, wenn wir $a$
kleiner machen, aber gleichzeitig $q$ vergrössern, so dass das Produkt
$qa$ konstant bleibt.
Man nennt $d=2qa$ das Dipolmoment der beiden Ladungen.
Der Dipol mit Dipolmoment $d$ hat daher für grosse $x$ das Potential
\[
f_2(x)
=
-\frac{d}{4\pi\varepsilon_0}\cdot \frac1{x^2}\cdot \frac1{1-\frac{a^2}{x^2}}
\simeq
-\frac{d}{4\pi\varepsilon_0}\cdot \frac1{x^2}.
\]

Aus diesen Beispielen können wir die Vermutung ableiten, dass weitere
Terme die Form
\[
-\frac{b}{4\pi\varepsilon_0}\cdot\frac1{x^k}
\]
haben müssen, wobei $b$ die Dimension einer Ladung mal die $(k-1)$-te
Potenz einer Distanz sein muss.

\subsection{Stetige Ladungsverteilung%
\label{skript:multipol:section:stetigeladungsverteilung}}
Betrachten wir jetzt statt zweier Ladungen eine beliebige 
Ladungsverteilung $\varrho(x)$, die aber ausserhalb des Intervalls
$[-a,a]$ verschwindet, also $\varrho(x)=0$ für $|x|>a$.
Stellen wir uns die Ladungsverteilung als eine Überlagerung einzelner
Ladungen an Positionen $y\in[-a,a]$ vor, dann k"onnen wir das
Potential des Fernfeldes sofort aus~\eqref{skript:multipol:2abfall}
ableiten:
\[
f(x)
\simeq
\underbrace{\int_{-a}^a \varrho(y)\,dy }_{\displaystyle =q}\cdot
\frac{1}{4\pi\varepsilon_0}\cdot\frac1x
+
\underbrace{\int_{-a}^a \varrho(y)y \,dy}_{\displaystyle =d}\cdot
\frac{1}{4\pi\varepsilon_0}\cdot\frac1{x^2}
+
\dots
=
\sum_{k=0}^\infty\int_{-a}^a\varrho(y)y^k\,dy\frac{1}{4\pi\varepsilon_0}\cdot\frac1{x^k} 
\]
Die Integrale 
\[
b_k=\int_{-a}^a \varrho(y)\,y^k\,dy
\]
heissen die $k$-ten Momente der Verteilung $\varrho$.
Die $k$-ten Momente bestimmen also die die Darstellung des Fernfeldes
einer Ladungsverteilung vollständig durch die Formel
\[
f(x)
\simeq
\frac1{4\pi\varepsilon_0}\sum_{k=0}^\infty b_k\frac{1}{x^k}.
\]
Die Summe auf der rechten Seite muss nicht mit dem Potential der
Ladungsverteilung übereinstimmen.
Vielmehr ist die Summe eine Zerlegung des Feldes in grosser Entfernen
nach ``Abfalls-Geschwindigkeit'',
ganz ähnlich wie die Fourier-Koeffizienten einer im Intervall
$[-\pi,\pi]$ definierten Funktion $f(x)$
\[
a_k=\frac1{2\pi}\int_{-\pi}^\pi f(x)\cos kx\,dx
\qquad\text{und}\qquad
b_k=\frac1{2\pi}\int_{-\pi}^\pi f(x)\sin kx\,dx
\]
eine Analyse nach Frequenzen bilden.
Die Fourierreihe
\[
\frac{a_0}2+\sum_{k=1}^\infty (a_k\cos kx+b_k\sin kx)
\]
ist eine periodische Funktion auf der ganzen reellen Achse, die
im Intervall $[-\pi,\pi]$ mit der ursprünglichen Funktion übereinstimmt.
Die Summe stimmt also mit der ursprünglichen Funktion nicht direkt
überein.

\subsection{Geometrische Reihe%
\label{skript:multipol:section:geometrischereihe}}
Bei der Analyse sowohl einer um $a$ verschobenen Einzelladung sowie auch
des Dipols aus zwei Ladungen bei $\pm a$ haben wir am Schluss Terme der
Form
\[
\frac1{1-\frac{a}{x}}
\qquad\text{bzw.}\qquad
\frac1{1-\frac{a^2}{x^2}}
\]
vernachlässigt.
Wir suchen nach einer Möglichkeit, diese Terme exakt zu berücksichtigen,
und trotzdem nur eine einfache Potenzreihe zu bekommen.

In der Analysis lernt man, die Summe der geometrische Reihe 
\[
s=a+aq+aq^2+\dots + aq^n
\]
zu berechnen.
Dazu bildet man $qs - s$ und findet
\begin{align*}
qs-s
s(q-1)
&=
q(a+aq+aq^2+\dots + aq^n)-(a+aq+aq^2+\dots + aq^n)
\\
&=
a(q+q^2+q^3\dots+q^{n+1}-1-q-q^2-\dots-q^n)
=
a(q^{n+1}-1)
\end{align*}
und schliesst
\[
s=a\frac{q^{n+1}-1}{q-1}.
\]
Wenn $|q|<1$ ist, dann kann man die Summe für beliebig grosse $n$ bilden,
und bekommt im Grenzwert $n\to\infty$
\begin{equation}
\sum_{k=0}^\infty aq^k = a\frac{1}{1-q}.
\label{skript:multipol:geom}
\end{equation}
Wir können diese Formel verwenden, um die bisher vernachlässigten Terme
exakt wiederzugeben:
\begin{align*}
\frac{1}{1-\frac{a}{x}}
&=
1+\frac{a}{x}+\frac{a^2}{x^2}+\dots
=
\sum_{k=0}^\infty \frac{a^k}{x^k}
\\
\frac{1}{1-\frac{a^2}{x^2}}
&=
1+\frac{a^2}{x^2}+\frac{a^4}{x^4}+\dots
=
\sum_{k=0}^\infty \frac{a^{2k}}{x^{2k}}
\\
\frac{1}{1+\frac{a^2}{x^2}}
&=
1-\frac{a^2}{x^2}+\frac{a^4}{x^4}-\dots
=
\sum_{k=0}^\infty (-1)^k\frac{a^{2k}}{x^{2k}}
\end{align*}
Die letzte Formel erhält man, indem man in \eqref{skript:multipol:geom}
$q$ durch $-q$ ersetzt.

Wir wenden dies jetzt wieder auf eine Ladungsverteilung
$\varrho(y)$ im Intervall $[-a,a]$ an.
Das Potential einer Einheitsladung an der Stelle $y$ ist
\[
f_y(x)
=
\frac1{4\pi\varepsilon_0}\cdot \frac{1}{x}\cdot\frac{1}{1-\frac{x}{y}}
=
\frac1{4\pi\varepsilon_0}\cdot \frac{1}{x}
\sum_{k=0}^\infty \frac{y^k}{x^k}
=
\frac1{4\pi\varepsilon_0}
\sum_{k=0}^\infty \frac{y^k}{x^{k+1}}.
\]
Durch Überlagerung mit Hilfe der Ladungsverteilung $\varrho(y)$ 
erhalten wir die exakte Formel für das Potential der Ladungsverteilung
\begin{equation}
f(x)
=
\int_{-a}^af_y(x)\,dy
=
\frac{1}{4\pi\varepsilon_0}
\sum_{k=0}^\infty
\underbrace{\int_{-a}^a\varrho(y)\,y^k\,dy}_{\displaystyle =b_k}
\cdot\frac{1}{x^{k+1}}.
=
\frac{1}{4\pi\varepsilon_0}
\sum_{k=0}^\infty b_k\frac{1}{x^{k+1}},
\label{skript:multipol:reiheexakte}
\end{equation}
wobei die $b_k$ wieder die $k$-ten Momente der Ladungsverteilung
$\varrho(y)$ sind.

Es stellt sich also heraus, dass die in
Abschnitt~\ref{skript:multipol:section:stetigeladungsverteilung}
erratene Entwicklung sogar eine exakte Darstellung ist.
Bricht man die Reihe~\eqref{skript:multipol:reiheexakte} nach zwei
Termen ab, erhält man die Monopol- und Dipol-Komponenten des
Fernfeldes.
Alle späteren Terme der Reihe beschreiben die feineren Strukturen
der Ladungsverteilung und fallen so rasch mit der Entfernung ab,
dass sie für Grosse Entfernungen vernachlässigt werde können.

Bei der Diskussion des Dipolmoments haben wir festgestellt, dass das
Fernfeld nur von der Grösse $d=2aq$ abhängt.
Verkleinert man den Abstand $a$ und vergrössert man gleichzeitig $q$,
so dass $d$ gleich gross bleibt, dann ändert sich das Fernfeld nicht.
Die Reihe~\eqref{skript:multipol:reiheexakte} verallgemeinert diese
Aussage auf eine beliebige Ladungsverteilung.
Ändern wir die Ladungsverteilung, achten aber darauf, dass die
$k$-ten Momente gleich bleiben, dann ändert sich das ausserhalb der
Ladungsverteilung messbare Potential nicht.
Die $k$-ten Momente beschreiben das ausserhalb der Ladungsverteilung
messbare Potential vollständig.

\subsection{Taylor-Reihe und Laurent-Reihe}
Im vorangegangenen Abschnitt haben wir gesehen, dass für grosse Werte von
$x$ die zur Diskussion stehende Funktion im Wesentlichen als Reihe
der inverse Potenzen $1/x^k$ beschrieben werden kann, also
\[
f(x)=\sum_{k=1}^\infty b_k\frac1{x^k}.
\]
Schreiben wir $z=1/x$, dann wird daraus eine gewöhnliche Potenzreihe
\[
\tilde f(z)=\sum_{k=1}^\infty b_k z^k.
\]
Wenn die Funktion $z\mapsto \tilde f(z)$ differenzierbar ist, dann
können die Koeffizienten $b_k$ können aus den Ableitungen der Funktion
$\tilde f$ gefunden
\begin{equation}
b_k=\frac{\tilde f^{(k)}(0)}{k!}
\qquad\Rightarrow\qquad
\tilde f(z)
=
\sum_{k=0}^\infty \frac{\tilde f^{(k)}(0)}{k!}z^k.
\end{equation}
Wir können dies betrachten als die Taylorreihe der Funktion $f$ um
den Punkt $\infty$.
\index{Taylor-Reihe}

Man nennt eine Reihe der Form
\[
\sum_{k=-\infty}^{\infty} (z-a)^k
\]
eine {\em Laurent-Reihe} im Punkt $z$.
\index{Laurent-Reihe}
Offensichtlich ist der Funktionswert in $z=a$ nicht definiert, und
oft wird die Reihe auch in einer Umgebung von $a$ nicht konvergieren.
Sie ist aber hervorragend dazu geeignet, das Verhalten einer Funktion
zu analysieren, die im Punkt $a$ eine Singularität aufweisen.
Besonders nützlich ist sie dann, wenn $z$ komplex ist, denn man
kann zeigen, dass jede komplex differenzierbare Funktion mit einer
Singularität im Punkt $a$ in einer Umgebung von $a$ als Laurent-Reihe
dargestellt werden kann.

Wenn wir also in der Lage sind, das eindimensionale Beispiel auf höhere
Dimensionen zu verallgemeinern, dann haben wir auch eine Verallgemeinerung
der Idee einer Taylorreihe um den Punkt $\infty$ auf höhere Dimensionen
gefunden.









%
% m-multipole.tex
%
% (c) 2017 Prof Dr Andreas Müller, Hochschule Rapperswil
%
\section{Multipolentwicklung}
\rhead{Multipolentwicklung}
In Abschnitt~\ref{skript:multipol:1dimbeispiel} haben wir ein
eindimensionales Beispiel untersucht, und mussten nach dem Potential
eines Dipols mit Vermutungen operieren, wie die weiteren Terme aussehen
müssten.
In diesem Abschnitt arbeiten wir in drei Dimensionen, und sind daher
in der Lage, kompliziertere Konfigurationen von Ladungen zu
konstruieren, und damit auch die späteren Terme der Entwicklung
genauer zu untersuchen.


\subsection{Dipol}
Wir betrachten das Feld eines Paares von entgegengesetzen Ladungen
in den Punkten $(0,0,a)$ und $(0,0,-a)$.
Der Einfachheit halber führen wir die Rechnung zunächst nur
in der $x$-$z$-Ebene durchführen und erst später mit Hilfe einer
vektoriellen Schreibweise auf drei Dimensionen erweitern.

Entlang der $z$-Achse kennen wir das Potential bereits aus
dem vorangegangenen Abschnitt.
Entlang der $x$-Achse verschwindet das Potential, denn die Punkte
auf der $x$-Achse sind von den beiden Ladungn gleich weit entfernt,
haben als entgegengesetzt gleiches Potential bezüglich beiden Ladungen
und damit totales Potential 0.

Wir betrachten jetzt das Potential im Punkt $(x,z)$, es ist
\[
f(x,z)
=
\frac{q}{45\pi\varepsilon_0}
\biggl(
\frac{1}{\sqrt{x^2 + (z-a)^2}}
-
\frac{1}{\sqrt{x^2 + (z+a)^2}}
\biggr)
\]
Offenbar müssen wir die Nenner besser verstehen, um diese Summe 
umformen zu können.
Speziell müssen wir die Abhängigkeit von der Entfernung
$r=\sqrt{x^2+z^2}$ vom Nullpunkt
und die Richtungsabhängigkeit voneinander trennen.
Dazu betrachten wir nur einen einzelnen Term
\[
\frac{1}{\sqrt{x^2+(z+a)^2}}
=
\frac{1}{\sqrt{x^2+z^2+2az+a^2}}
=
\frac{1}{\sqrt{x^2+z^2}} \cdot \frac{1}{\sqrt{1+\frac{2az+a^2}{x^2+z^2}}}
=
\frac{1}{r} \cdot \frac{1}{\sqrt{1+\frac{2az+a^2}{r^2}}}
\]
Offenbar ist es notwendig, einen Term der Form
\[
\frac{1}{\sqrt{1+t}}
=
(1+t)^{-\frac12}
\]
besser zu verstehen.
Die Taylor-Reihe kann für eine allgemeinere Art von Reihe bestimmt werden,
nämlich $g(t)=(1+t)^\alpha$.
Dazu müssen die Ableitungen bestimmt werden
\begin{align*}
g'(t)
&=
\alpha(1+t)^{\alpha-1}
&
g'(0)&=\alpha
\\
g''(t)
&=
\alpha(\alpha-1)(1+t)^{\alpha-2}
&
g''(0)&=\alpha(\alpha-1)
\\
&\;\vdots
\\
g^{(n)}(t)
&=
\alpha(\alpha-1)(\alpha-2)\dots(\alpha-n+1) (1+t)^{\alpha -n}
&
g^{(n)}(0)&=\alpha(\alpha-1)(\alpha-2)\dots(\alpha -n +1)
\end{align*}
Der Term zur Potenz $n$ ist
\[
\frac{\alpha(\alpha-1)(\alpha-2)\dots(\alpha-n+1)}{n!} t^n
\]
Der Bruch sieht genau so aus wie der Binomialkoeffizient, daher schreibt
man auch für nicht ganzzahlige $\alpha$
\[
\binom{\alpha}{n}
=
\frac{\alpha(\alpha-1)(\alpha-2)\dots(\alpha-n+1)}{n!}.
\]
Mit dieser Schreibweise bekommen wir die Taylorreihe
\[
(1+t)^\alpha=\sum_{k=1}^\infty \binom{\alpha}{k} t^k
\]
für die Funktion $(1+t)^\alpha$.
Diese Reihe ist konvergent für $|t|<1$.

Wir interessieren uns speziell f"ur den Fälle $\alpha=\pm\frac12$.
Wir betrachten erst den Fall $\alpha=-\frac12$.
Der Term zur Potenz $n$ ist
\[
\frac{
\bigl(-\frac12\bigr)
\bigl(-\frac32\bigr)
\bigl(-\frac52\bigr)
\cdots
\bigl(-\frac{2n+1}2\bigr)}{n!} t^n
=
(-1)^n \frac{1\cdot 3\cdot 5 \cdots (2n + 1)}{2^n\cdot 1\cdot 2\cdot 3\cdots n}
=
(-1)^n \frac{1\cdot 3\cdot 5 \cdots (2n+1)}{2\cdot 4\cdot 6\cdots (2n)}
\]
Die zugehörige Potenzreihe ist daher
\[
\frac1{\sqrt{1+t}}
=
1-\frac12t+\frac3{8}t^2-\frac{15}{48}t^3+\frac{105}{354}t^4-\dots
\]
F"ur 
\[
\frac1{\sqrt{1+t}}
=
1-\frac12 t
+ \frac12\frac32\frac12 t^2
- \frac 12\frac32\frac 52\frac1{3!} t^3
+ \frac 12\frac32\frac 52\frac72\frac1{4!} t^4
- \frac 12\frac32\frac 52\frac72\frac92\frac1{5!} t^4
+\dots
\]



%
% m-kugelfunktionen.tex
%
% (c) 2017 Prof Dr Andreas Müller, Hochschule Rapperswil
%
\section{Kugelfunktionen}
\rhead{Kugelfunktionen}






