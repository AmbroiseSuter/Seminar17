%
% k-gravitatzion.tex
%
% (c) 2017 Prof Dr Andreas Müller, Hochschule Rapperswil
%
\section{Gravitation%
\label{skript:kruemmung:sectipn:gravitation}}
Albert Einstein hat erkannt, dass die Wirkung der Gravitation 
durch die Krümmung des Raumes beschrieben werden muss.

\subsection{Physikalische Bedeutung der Krümmung}

\subsection{Einstein-Tensor}

\subsection{Feldgleichungen}

%
% k-schwarzschild.tex
%
% (c) 2017 Prof Dr Andreas Müller, Hochschule Rapperswil
%

\subsection{Schwarzschild-Metrik}
Die Feldgleichungen schränken die Metriken ein, die in einem Raumzeit-Gebiet
überhaupt möglich sind.
Damit stellt sich automatisch die Frage, wie die Einsteinsche Theorie 
das Gravitationsfeld in der Umgegung eines Sterns beschreiben kann.
Schon wenige Monate nachdem Einstein seine allgemeine Relativitätstheorie
veröffentlich hat, hat Karl Schwarzschild eine Lösung der Einsteinschen
Feldgleichungen gefunden.
Daraus lassen sich die Bewegungsgleichungen eines Körpers in der Nähe
eines Sterns ableiten und es sollte möglich sein, den Unterschied zwischen
der Newtonschen Gravitations-Theorie und Einsteinschen  zu quantifizieren
und damit Tests der allgmeinen Relativitätstheorie zu ermöglichen.

\subsubsection{Eine kugelsymmetrische, statische Lösung}
Karl Schwarzschild suchte eine Metrik, die sich mit der Zeit nicht ändert,
also
\[
\frac{\partial g_{\mu\nu}}{\partial t}=0,
\]
und die ausserdem kugelsymmetrisch sein soll.
Diese Metrik sollte eine erste Approximation für die Gravitation in
der Umgebung eines Sterns sein.
Natürlich berücksichtigt dieses Modell weder, dass sich Sterne mit
der Zeit entwickeln, noch die Tatsache, dass sich Sterne normalerweise
um eine Achse drehen, dass man also gar nicht eine rotationssymmetrische
Lösung erwarten darf.

Die Längenmessung
\begin{equation}
ds^2
=
-c^2\,dt^2 + dr^2 + r^2 d\Omega^2
\qquad
d\Omega^2 = d\vartheta^2 + \sin^2\vartheta\,d\varphi^2
\label{skript:kruemmung:euklid}
\end{equation}
im Raum mit Kugelkoordinaten ist natürlich eine solche Metrik,
doch da dies nur eine andere Parametrisierung des euklidischen
Raumes ist, ist diese Geometrie flach.
Sie kann also sich nicht ein Modell eines Sternes sein.

Man kann eine Lösung der Feldgleichungen finden, indem man den
einzelnen Termen der euklischen Metrik~\eqref{skript:kruemmung:euklid}
zunächst unbestimmte Faktoren hinzufügt, die nur von $r$ abhängt,
und dann mit Hilfe der Feldgleichungen dafür Differentialgleichungen
herleitet.
Wir wollen diesen beschwerlichen Weg nicht gehen, und uns mit dem
Resultate zufriedenstellen, es lautet
\begin{equation}
ds^2
=
-\biggl(1-\frac{r_g}r\biggr)c^2\,dt^2
+\frac1{\displaystyle 1-\frac{r_g}r}\,dr^2 + r^2\,d\Omega^2.
\label{skript:kruemmung:schwarzschildmetrik}
\end{equation}
Man kann nachrechnen, zum Beispiel mit Hilfe der früher vorgestellen
Maxima-Programme, dass der Einstein-Tensor für diese Metrik überall
verschwindet.

\subsubsection{Der Fall $r=r_g$}
Es scheint, dass die Metrik für $r=r_g$ nicht wohldefiniert ist,
da dann der Nenner im zweiten Term nicht wohldefiniert ist.
Dem ist jedoch nicht so, wie man durch Wahl eines anderen Koordinatensystems
zeigen kann.
Wir ersetzen die Zeitkoordinaten $t$ durch $\tau$, und die $r$-Koordinate
durch $R$ mit der vorläufig noch unbestimmten Funktion $f(r)$.
Es soll gelten
\begin{equation}
\begin{aligned}
c\tau
&=
ct + \int\frac{f(r)\,dr}{\displaystyle 1-\frac{r_g}{r}}
&
&\text{und}
&
R
&=
ct
+
\int\frac{dr}{\displaystyle \biggl(1-\frac{r_g}{r}\biggr)f(r)}
\end{aligned}
\label{skript:kruemmung:finkelsteindefinition}
\end{equation}
mit einer vorläufig noch unbestimmten Funktion $f(r)$.
Die Integrationskonstante in den beiden unbestimmten Integralen
entspricht einer Wahl des $t$- bzw.~$\tau$-Nullpunktes und ist
daher nicht von Bedeutung.
Um die Metrik in $\tau$ und $R$ ausdrücken zu können, müssen wir 
den Zusammenhang zwischen $dr$, $dR$, $dt$ und $d\tau$ kennen.
Wir finden diesen durch Ableiten der beiden Definitionsgleichungen
\eqref{skript:kruemmung:finkelsteindefinition}:
\begin{align*}
c\,d\tau
&=
c\,dt + \frac{f(r)}{\displaystyle 1-\frac{r_g}{r}}\,dr,
\\
dR
&=
c\,dt
+
\frac{1}{\displaystyle\biggl(1-\frac{r_g}{r}\biggr)f(r)}\,dr.
\end{align*}
Für die Schwarzschild-Metrik brauchen wir die Quadrate davon:
\begin{align*}
c^2\,d\tau ^2
&=
c^2\,dt^2 + 2\frac{cf(r)}{\displaystyle 1-\frac{r_g}{r}}\,dt\,dr
+\frac{f(r)^2}{\biggl(\displaystyle 1-\frac{r_g}{r}\biggr)^2}\,dr^2,
\\
dR^2
&=
c^2\,dt^2 + 2\frac{c}{\displaystyle\biggl(1-\frac{r_g}{r}\biggr)f(r)}\,dt\,dr
+
\frac{1}{\displaystyle \biggl(1-\frac{r_g}{r}\biggr)^2f(r)^2}\,dr^2.
\end{align*}
Wir müssen diese beiden Ausdrücke so kombinieren, dass der gemischte
Term $dt\,dr$ wegfällt, denn dieser kommt in der Schwarzschild-Metrik
nicht vor.
Dazu multiplizieren wir die zweite Zeile mit $f(r)^2$, und subtrahieren.
Wir erhalten
\begin{equation}
c^2\,d\tau^2 - f(r)^2\,dR^2
=
(1-f(r)^2)c^2\,dt^2
+\frac{f(r)^2-1}{\displaystyle\biggl(1-\frac{r_g}{r}\biggr)^2}\,dr^2.
\end{equation}
Damit daraus die Schwarzschild-Metrik wird, müssen wir vor dem $dt^2$
Term einen Faktor der Form $(1-r_g/r)$ haben, wir müssen also zunächst
alles durch $1-f(r)^2$ dividieren, und dann mit $1-r_g/r$ multiplizieren.
Wir kehren ausserdem das Vorzeichen und erhalten
\[
\frac{\displaystyle 1-\frac{r_g}{r}}{1-f(r)^2}
(-c^2\,d\tau^2 + f(r)^2\,dR^2)
=
-\biggl(1-\frac{r_g}{r}\biggr)c^2\,dt^2
+\frac{1}{\displaystyle 1-\frac{r_g}{r}}\,dr^2
\]
Bis auf die Terme $d\Omega^2$ steht auf der rechten Seite die
Schwarzschild-Metrik.
Von den Termen auf der linken Seite ist nur der Bruch
\[
\frac{\displaystyle 1-\frac{r_g}{r}}{1-f(r)^2}
\]
problematisch, allerdings nur, wenn der Nenner $1-f(r)^2$ eine
Nullstelle hat.
Wir können den ganzen Bruch zum Verschwinden bringen, indem wir die
Wahlmöglichkeiten für $f(r)$ ausnützen und 
\[
f(r)=\sqrt{\frac{r_g}{r}}
\]
wählen.
Setzen wir dies ein, erhalten wir die Metrik in $\tau$-$R$-Koordinaten
jetzt in der Form
\begin{equation}
ds^2
=
-c^2\,d\tau^2 + \frac{r_g}{r}\,dR^2 + r^2 d\Omega^2,
\label{skript:kruemmung:finkelstein}
\end{equation}
aber natürlich müssen wir $r$ und $r^2$ ebenfalls durch die Koordinaten
$\tau$ und $R$ ausdrücken.

Da jetzt $f(r)$ bekannt ist, können wir die
Integrale~\eqref{skript:kruemmung:finkelsteindefinition} auswerten.
Wir berechnen
\begin{align*}
R-c\tau
&=
\int\frac{dr}{\displaystyle \biggl(1-\frac{r_g}{r}\biggr)f(r)}
- \int\frac{f(r)\,dr}{\displaystyle 1-\frac{r_g}{r}}
=
\int\frac{1-f(r)^2}{\displaystyle\biggl(1-\frac{r_g}{r}\biggr) f(r)}\,dr
=
\int\frac{1}{f(r)}\,dr
=
\frac{1}{\sqrt{\mathstrut r_g}}\int\sqrt{\mathstrut r}\,dr
\\
&=
\frac{1}{\sqrt{\mathstrut r_g}} \frac{2}{3}r^{\frac{3}{2}}
\end{align*}
Damit kann man jetzt $r$ durch $R-c\tau$ ausdrücken
\begin{equation}
r
=
\biggl(\frac{3}{2}(R-c\tau)r_g^{\frac{1}{2}}\biggr)^{\frac23}
=
\biggl(\frac{3}{2}(R-c\tau)\biggr)^{\frac23} r_g^{\frac13}
\label{skript:kruemmung:finkelsteinr}
\end{equation}
Der $r$-Koordinate $r_g$ entspricht jetzt die Gerade
\[
R-c\tau = \frac23 r_g = R_g.
\]
Setzen wir dies in die Metrik~\eqref{skript:kruemmung:finkelstein}
ein, erhalten wir
\[
ds^2
=
-c^2 \,d\tau^2
+r_g^{\frac23}\biggl(\frac32(R-c\tau)\biggr)^{-\frac23}\,dR^2
+ r_g^{\frac23}\biggl(\frac32(R-c\tau)\biggr)^{\frac43}\,d\Omega^2.
\]
Es ist klar, dass in diesen Koordinaten in keinem Term für $r=r_g$
eine Singularität auftritt.
Die hier berechneten Koordinaten $\tau$ und $R$ heissen
Finkelstein-Koordinaten.

\subsubsection{Absturz ins Zentrum}
In Finkelstein-Koordinaten verschwindet die Singularität bei $r=r_g$,
es sollte daher auch einfacher werden, die Bahn eines radial ins
Zentrum des Feldes stürzenden Teilchens zu berechnen.
Dazu können wir die Winkel-Koordinaten $\vartheta$ und $\varphi$
vernachlässigen, und nur mit den Koordinaten $\tau$ und $R$ und der
Metrik
\[
ds^2
=
-c^2 \,d\tau^2
+r _g^{\frac23}\biggl(\frac32(R-c\tau)\biggr)^{\frac23}\,dR^2
\]
arbeiten.
Zur Berechnung der Bahnkurven brauchen wir die Christoffel-Symbole
2.~Art.
Die Berechnung mit Maxima ergibt
\begin{align*}
\Gamma^1_{11} &= 0,&
\Gamma^1_{12} &= 0,&
\Gamma^1_{21} &= 0,&
\Gamma^1_{22} &= \frac{2^{\frac23}r_g^{\frac23}}{(3(R-c\tau))^{\frac53}},\\
\Gamma^2_{11} &= 0,&
\Gamma^2_{12} &= \frac1{3(R-c\tau)},&
\Gamma^2_{21} &= \frac1{3(R-c\tau)},&
\Gamma^2_{22} &= -\frac1{3(R-c\tau)}.
\end{align*}
Wir beschreiben die Bahn eines Teilchens welches sich in diesem Feld
radial bewegt mit den Funktionen $\tau(s)$ und $R(s)$.
Wir untersuchen ein Teilchen, welches beim Punkt $\tau(0)=0$ und $R(0)=R_0$
mit Anfangsrichtung $\dot\tau(0)=1$ und $\dot R(0)=0$ in das Gravitationsfeld
stürzt.
Die Geodätengleichungen lauten
\begin{align*}
\frac{d^2\tau(s)}{ds^2}
&=
\Gamma^1_{22}\biggl(\frac{dR(s)}{ds}\biggr)^2,
\\
\frac{d^2R(s)}{ds^2}
&=
\Gamma^2_{12}\frac{d\tau(s)}{ds}\frac{dR(s)}{ds}
+
\Gamma^2_{22}\biggl(\frac{dR(s)}{ds}\biggr)^2.
\end{align*}
Da für die Anfangsbedinung $\dot R=0$ ist und auf der rechten Seite
$\dot R$ in jedem Term vorkommt, folgt aus der zweiten Gleichung
$\ddot R=0$, $\dot R$ kann sich also nicht ändern.
Wenn aber $\dot R=0$ für alle Werte von $s$, dann ist nach der ersten
Gleichung auch $\ddot \tau=0$ und damit $\dot \tau=1$.
Die Lösungskurve ist also $R=R_0$ und $\tau=s$.

Mit Hilfe der Formel~\eqref{skript:kruemmung:finkelsteinr}
kann man damit auch $r$ für den radialen Absturz angeben:
\[
r(\tau)=\biggl(\frac32(R_0-c\tau)\biggr)^{\frac23}r_g^{\frac13}.
\]

\subsubsection{Ereignishorizont}
Wir betrachten wieder die
Schwarzschild-Metrik~\eqref{skript:kruemmung:schwarzschildmetrik}
und möchten genauer verstehen, was bei $r=r_g$ passiert.
Dazu überlegen wir uns, wie ein Geodäte eines realen Teilchens 
aussieht.
Wir wissen, dass reale Teilchen sich so bewegen müssen, dass der
Tangentialvektor negative Metrik hat.
Ein Teilchen in Ruhe hat zum Beispiel $\dot r=0$, $\dot \vartheta=0$ und
$\dot varphi=0$, einzig $\dot t\ne 0$.
Da der Koeffizient von $dt^2$
\[
-\biggl(1-\frac{r_g}{r}\biggr)
\quad
\begin{cases}
<0&\qquad r > r_g
\\
>0&\qquad r < r_g
\end{cases}
\]
für $r>r_g$
negativ ist folgt, dass es tatsächlich möglich ist, dass sich ein Teilchen
in festem Abstand vom Zentrum aufhalten kann.
Sobald aber $r<r_g$ ist, ist der Koeffizient von $dt^2$ positiv,
fester Abstand vom Zentrum ist nicht mehr möglich.

Für $r<r_g$ wechselt aber auch das Vorzeichen des Koeffizienten
\[
\frac1{\displaystyle1-\frac{r_g}{r}}
\quad
\begin{cases}
>0&\qquad r>r_g
\\
<0&\qquad r<r_g
\end{cases}
\]
von $dr^2$, damit wird plötzlich der Vektor $\dot t=0$, $\dot r\ne 0$,
$\dot\vartheta=\dot\varphi=0$ zeitartig.
Sobald ein Teilchen sich innerhalb des Radius $r_g$ befindet, kann
sein Radius nur noch abnehmen.
Die Berechnung der Bahn eines solchen Teilchens zeigt auch, dass
es unausweichlich in endlicher Zeit ins Zentrum stürzen wird.



\subsubsection{Was spürt man am Ereignishorizont?}
Der Riemannsche Krümmungstensor gibt an, wie schnell Geodäten von Teilchen
wegen unterschiedlicher Gravitationswirkung sich voneinander entfernen.
Er bestimmt also die Gezeitenkräfte, die ein Astronaut spürt, der in
diesem Gravitationsfeld abstürzt.
die Gezeitenkräfte, die ein Astronaut spürt, der in
diesem Gravitationsfeld abstürzt.
Die Berechnung des Riemann-Tensors mit Maxima ergibt aber:
\begin{align*}
R_{1212}&=-\frac{r_g}{r^3},
\end{align*}
die Gezeitenkräfte in radialer Richtung werden für den Astronauten
also zwar immer grösser, aber bei $r=r_g$ spürt er nichts besonderes.



\subsection{Robertson-Walker-Metrik}
Die Robertson-Walker-Metrik wurde entwickelt, um ein homogenes,
isotropes und expandierendes Universum zu modellieren.
Wir beginnen mit einem flachen Universum mit der
Metrik
\[
ds^2
=
-c^2\,dt^2 + dr^2 + r^2\,d\vartheta^2 + r^2\sin^2\vartheta \,d\varphi^2
=
-c^2\,dt^2 + dr^2 + d\Omega^2.
\]
Wenn dieses Universum expandiert, ändert sich die Längenmessung für die
Raum-Koordinaten, nicht aber für die Zeitkoordinate.
Wir können dies mit einem zeitabhängigen Skalierungsfaktor $a(t)$ 
modellieren:
\[
ds^2
=
-c^2\,dt^2 + a(t)\,dr^2 + a(t)r^2\,d\vartheta^2 + a(t)r^2 \sin^2\vartheta\,d\varphi^2
=
-c^2\,dt^2 + a(t)\,dr^2 + a(t)r^2\,d\Omega^2.
\]
Natürlich bleibt dies ein flaches Universum.

Wir müssen aber auch zulassen, dass das Universum gekrümmt ist.
Dies bedeutet, dass der Umfang eines Kreises nicht um den Nullpunkt
nicht proportional mit $r$ wächst.
Alternativ kann man auch sagen, dass die $r$-Koorindate eines Punktes 
eines Kreises vom Umfang $2\pi R$ nicht unbedingt $R$ sein muss, also
$r \ne R$.
Wir müssen also den Term $dr^2$ ersetzen durch etwas, was mit zunehmendem
$r$ grösser oder kleiner als $1$ sein wird.
\[
ds^2
=
-c^2\,dt^2
+ a(t)^2 \biggl(
\frac{R^2}{R^2-kr^2} dr^2
+
r^2\, d\Omega^2
\biggr)
\]

Der Faktor $a(t)$ heisst der Skalierungsfaktor, er beschreibt, wie stark
das Universum sich zur Zeit $t$ bereits gestreckt hat.

Der Einstein-Tensor dieser Metrik kann in einer sehr langwierigen oder
noch besser mechanischen Rechnung ermittelt werden.
Er ist diagonal und hat die folgenden Diagonal-Elemente:
\begin{align*}
G_{00}
&=
\frac{3}{4}\biggl(\frac{\dot a(t)}{a(t)}\biggr)^2
\\
G_{11}
&=
-\ddot a(t) +\frac{\dot a(t)^2}{4a(t)}
\\
G_{22}
&=
G_{11} r^2
\\
G_{33}
&=
G_{11} r^2\sin^2\vartheta
\end{align*}
In den Ausdrücken für $G_{22}$ und $G_{33}$ fällt auf, dass sie gegenüber
$G_{11}$ nur die zusätzlichen Faktoren enthalten, die auch in der Definition
der Robertson-Walker-Metrik evident sind.
Um die Feldgleichungen aufzustellen reicht es daher, mit $G_{00}$ und
$G_{11}$ zu arbeiten.
Dies wird später auf die Friedmann-Gleichungen führen.



