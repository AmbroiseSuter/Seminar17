%
% m-multipole.tex
%
% (c) 2017 Prof Dr Andreas Müller, Hochschule Rapperswil
%
\section{Multipolentwicklung}
\rhead{Multipolentwicklung}
In Abschnitt~\ref{skript:multipol:1dimbeispiel} haben wir ein
eindimensionales Beispiel untersucht, und mussten nach dem Potential
eines Dipols mit Vermutungen operieren, wie die weiteren Terme aussehen
müssten.
In diesem Abschnitt arbeiten wir in drei Dimensionen, und sind daher
in der Lage, kompliziertere Konfigurationen von Ladungen zu
konstruieren, und damit auch die späteren Terme der Entwicklung
genauer zu untersuchen.


\subsection{Dipol}
Wir betrachten das Feld eines Paares von entgegengesetzen Ladungen
in den Punkten $(0,0,a)$ und $(0,0,-a)$.
Der Einfachheit halber führen wir die Rechnung zunächst nur
in der $x$-$z$-Ebene durchführen und erst später mit Hilfe einer
vektoriellen Schreibweise auf drei Dimensionen erweitern.

Entlang der $z$-Achse kennen wir das Potential bereits aus
dem vorangegangenen Abschnitt.
Entlang der $x$-Achse verschwindet das Potential, denn die Punkte
auf der $x$-Achse sind von den beiden Ladungn gleich weit entfernt,
haben als entgegengesetzt gleiches Potential bezüglich beiden Ladungen
und damit totales Potential 0.

Wir betrachten jetzt das Potential im Punkt $(x,z)$, es ist
\[
f(x,z)
=
\frac{q}{4\pi\varepsilon_0}
\biggl(
\frac{1}{\sqrt{x^2 + (z-a)^2}}
-
\frac{1}{\sqrt{x^2 + (z+a)^2}}
\biggr)
\]
Offenbar müssen wir die Nenner besser verstehen, um diese Summe 
umformen zu können.
Speziell müssen wir die Abhängigkeit von der Entfernung
$r=\sqrt{x^2+z^2}$ vom Nullpunkt
und die Richtungsabhängigkeit voneinander trennen.
Dazu betrachten wir nur einen einzelnen Term
\[
\frac{1}{\sqrt{x^2+(z+a)^2}}
=
\frac{1}{\sqrt{x^2+z^2+2az+a^2}}
=
\frac{1}{\sqrt{x^2+z^2}} \cdot \frac{1}{\sqrt{1+\frac{2az+a^2}{x^2+z^2}}}
=
\frac{1}{r} \cdot \frac{1}{\sqrt{1+\frac{2az+a^2}{r^2}}}
\]
Offenbar ist es notwendig, einen Term der Form
\[
\frac{1}{\sqrt{1+t}}
=
(1+t)^{-\frac12}
\]
besser zu verstehen.
Die Taylor-Reihe kann für eine allgemeinere Art von Reihe bestimmt werden,
nämlich $g(t)=(1+t)^\alpha$.
Dazu müssen die Ableitungen bestimmt werden
\begin{align*}
g'(t)
&=
\alpha(1+t)^{\alpha-1}
&
g'(0)&=\alpha
\\
g''(t)
&=
\alpha(\alpha-1)(1+t)^{\alpha-2}
&
g''(0)&=\alpha(\alpha-1)
\\
&\;\vdots
\\
g^{(n)}(t)
&=
\alpha(\alpha-1)(\alpha-2)\dots(\alpha-n+1) (1+t)^{\alpha -n}
&
g^{(n)}(0)&=\alpha(\alpha-1)(\alpha-2)\dots(\alpha -n +1)
\end{align*}
Der Term zur Potenz $n$ ist
\[
\frac{\alpha(\alpha-1)(\alpha-2)\dots(\alpha-n+1)}{n!} t^n
\]
Der Bruch sieht genau so aus wie der Binomialkoeffizient, daher schreibt
man auch für nicht ganzzahlige $\alpha$
\[
\binom{\alpha}{n}
=
\frac{\alpha(\alpha-1)(\alpha-2)\dots(\alpha-n+1)}{n!}.
\]
Mit dieser Schreibweise bekommen wir die Taylorreihe
\[
(1+t)^\alpha=\sum_{k=1}^\infty \binom{\alpha}{k} t^k
\]
für die Funktion $(1+t)^\alpha$.
Diese Reihe ist konvergent für $|t|<1$.

Wir interessieren uns speziell f"ur den Fälle $\alpha=\pm\frac12$.
Wir betrachten erst den Fall $\alpha=-\frac12$.
Der Term zur Potenz $n$ ist
\[
\frac{
\bigl(-\frac12\bigr)
\bigl(-\frac32\bigr)
\bigl(-\frac52\bigr)
\cdots
\bigl(-\frac{2n+1}2\bigr)}{n!} t^n
=
(-1)^n \frac{1\cdot 3\cdot 5 \cdots (2n + 1)}{2^n\cdot 1\cdot 2\cdot 3\cdots n}
=
(-1)^n \frac{1\cdot 3\cdot 5 \cdots (2n+1)}{2\cdot 4\cdot 6\cdots (2n)}
\]
Die zugehörige Potenzreihe ist daher
\[
\frac1{\sqrt{1+t}}
=
1-\frac12t+\frac3{8}t^2-\frac{15}{48}t^3+\frac{105}{354}t^4-\dots
\]
F"ur 
\[
\frac1{\sqrt{1+t}}
=
1-\frac12 t
+ \frac12\frac32\frac12 t^2
- \frac 12\frac32\frac 52\frac1{3!} t^3
+ \frac 12\frac32\frac 52\frac72\frac1{4!} t^4
- \frac 12\frac32\frac 52\frac72\frac92\frac1{5!} t^4
+\dots
\]


