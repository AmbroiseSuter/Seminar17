%
% m-multipole.tex
%
% (c) 2017 Prof Dr Andreas Müller, Hochschule Rapperswil
%
\section{Multipolentwicklung}
\rhead{Multipolentwicklung}
In Abschnitt~\ref{skript:multipol:1dimbeispiel} haben wir ein
eindimensionales Beispiel untersucht, und mussten nach dem Potential
eines Dipols mit Vermutungen operieren, wie die weiteren Terme aussehen
müssten.
In diesem Abschnitt arbeiten wir in drei Dimensionen, und sind daher
in der Lage, kompliziertere Konfigurationen von Ladungen zu
konstruieren, und damit auch die späteren Terme der Entwicklung
genauer zu untersuchen.


\subsection{Dipol}
\begin{figure}
\centering
\includegraphics{chapters/tikz/dipol2.pdf}
\caption{Berechnung des Dipolpotentials 
\eqref{skript:multipol:dipol} in der $x$-$z$-Ebene
\label{skript:multipol:figure:dipol}}
\end{figure}
Wir betrachten das Feld eines Paares von entgegengesetzen Ladungen
in den Punkten $(0,0,a)$ und $(0,0,-a)$
(Abbildung~\ref{skript:multipol:figure:dipol}).
Der Einfachheit halber führen wir die Rechnung zunächst nur
in der $x$-$z$-Ebene durchführen und erst später mit Hilfe einer
vektoriellen Schreibweise auf drei Dimensionen erweitern.

Entlang der $z$-Achse kennen wir das Potential bereits aus
dem vorangegangenen Abschnitt.
Entlang der $x$-Achse verschwindet das Potential, denn die Punkte
auf der $x$-Achse sind von den beiden Ladungn gleich weit entfernt,
haben als entgegengesetzt gleiches Potential bezüglich beiden Ladungen
und damit totales Potential 0.

Wir betrachten jetzt das Potential im Punkt $(x,z)$, es ist
\begin{equation}
f(x,z)
=
\frac{q}{4\pi\varepsilon_0}
\biggl(
\frac{1}{\sqrt{x^2 + (z-a)^2}}
-
\frac{1}{\sqrt{x^2 + (z+a)^2}}
\biggr)
\label{skript:multipol:dipol}
\end{equation}
Offenbar müssen wir die Nenner besser verstehen, um diese Summe 
umformen zu können.
Speziell müssen wir die Abhängigkeit von der Entfernung
$r=\sqrt{x^2+z^2}$ vom Nullpunkt
und die Richtungsabhängigkeit voneinander trennen.
Dazu betrachten wir nur einen einzelnen Term
\[
\frac{1}{\sqrt{x^2+(z\pm a)^2}}
=
\frac{1}{\sqrt{x^2+z^2\pm 2az+a^2}}
=
\frac{1}{\sqrt{x^2+z^2}} \cdot \frac{1}{\sqrt{1+\frac{\pm 2az+a^2}{x^2+z^2}}}
=
\frac{1}{r} \cdot \frac{1}{\sqrt{1+\frac{\pm 2az+a^2}{r^2}}}
\]
Für grosse Werte von $r$ verschwindet der zweite Term in der Wurzel,
in erster Näherung verhält sich die Funktion daher wie $1/r$.
In den zwei Termen von \eqref{skript:multipol:dipol} hebt sich dieses
Verhalten jedoch weg, um die Funktion $f(x,z)$ zu verstehen, ist es
daher nötig, die Abweichungen von $1/r$ genauer zu verstehen.

Offenbar müssen wir Terme der Form
\begin{equation}
\frac{1}{\sqrt{1+t}}
=
(1+t)^{-\frac12}
\label{skript:multipol:wurzel}
\end{equation}
ausrechnen können, wobei wir später $t=(\pm2az+a^2)/r^2$ setzen wollen.

\subsubsection{Binomialreihe}
Die Taylor-Reihe der Funktion \eqref{skript:multipol:wurzel}
kann für eine allgemeinere für die Funktionen
\begin{equation}
g(t)=(1+t)^\alpha
\end{equation}
mit beliebigem Exponenten bestimmt werden.

Dazu müssen die Ableitungen der Funktion $g(t)$ bestimmt werden
\begin{align*}
g'(t)
&=
\alpha(1+t)^{\alpha-1}
&
g'(0)&=\alpha
\\
g''(t)
&=
\alpha(\alpha-1)(1+t)^{\alpha-2}
&
g''(0)&=\alpha(\alpha-1)
\\
&\;\vdots
\\
g^{(n)}(t)
&=
\alpha(\alpha-1)(\alpha-2)\dots(\alpha-n+1) (1+t)^{\alpha -n}
&
g^{(n)}(0)&=\alpha(\alpha-1)(\alpha-2)\dots(\alpha -n +1)
\end{align*}
Der Term zur Potenz $n$ in der Taylor-Reihe von $g(t)$ ist
\[
\frac{\alpha(\alpha-1)(\alpha-2)\dots(\alpha-n+1)}{n!} t^n
\]
Wäre $\alpha$ eine ganze Zahl, dann sähe der Bruch
genau so aus wie der Binomialkoeffizient $\binom{\alpha}{k}$.
In Erweiterung der üblichen Definition des Binomialkoeffizienten
schreibt man auch für nicht ganzzahlige $\alpha$
\[
\binom{\alpha}{n}
=
\frac{\alpha(\alpha-1)(\alpha-2)\dots(\alpha-n+1)}{n!}.
\]
Mit dieser Schreibweise bekommen wir die Taylorreihe
\[
(1+t)^\alpha=\sum_{k=1}^\infty \binom{\alpha}{k} t^k
\]
für die Funktion $(1+t)^\alpha$.
Sie heisst die {\em Binomialreihe} und ist konvergent für $|t|<1$.
\index{Binomialreihe}

\subsubsection{Der Fall $\alpha=-\frac12$}
Wir betrachten die Binomialreihe für den Fall $\alpha=-\frac12$.
Der Term zur Potenz $n$ ist
\begin{equation}
\frac{
\bigl(-\frac12\bigr)
\bigl(-\frac32\bigr)
\bigl(-\frac52\bigr)
\cdots
\bigl(-\frac{2n+1}2\bigr)}{n!} t^n
=
(-1)^n \frac{1\cdot 3\cdot 5 \cdots (2n + 1)}{2^n\cdot 1\cdot 2\cdot 3\cdots n}
=
(-1)^n \frac{1\cdot 3\cdot 5 \cdots (2n+1)}{2\cdot 4\cdot 6\cdots (2n)}
\label{skript:multipol:koeffizienten}
\end{equation}
Die zugehörige Potenzreihe ist daher
\[
\frac1{\sqrt{1+t}}
=
1-\frac12t+\frac3{8}t^2-\frac{15}{48}t^3+\frac{105}{354}t^4-\dots
\]
%deren Koeffizienten man auch faktorisieren und damit etwas
%Für 
%\[
%\frac1{\sqrt{1+t}}
%=
%1-\frac12 t
%+ \frac12\frac32\frac12 t^2
%- \frac 12\frac32\frac 52\frac1{3!} t^3
%+ \frac 12\frac32\frac 52\frac72\frac1{4!} t^4
%- \frac 12\frac32\frac 52\frac72\frac92\frac1{5!} t^4
%+\dots
%\]

Wir verwenden die binomische Reihe jetzt, um das Dipolpotential
zu berechnen.
Setzen wir 
\[
t=\frac{\pm 2az+a^2}{r^2}
\]
in der binomischen Reihe, erhalten wir
\[
\frac{1}{\sqrt{1+\frac{\pm 2az+a^2}{r^2}}}
=
1-\frac12\frac{\pm 2az+a^2}{r^2}
+
\frac38 \biggl(\frac{\pm 2az+a^2}{r^2}\biggr)^2+\dots
\]
und für das Dipolpotential $f(x,z)$ gemäss \eqref{skript:multipol:wurzel}
\begin{align*}
f(x,z)
&=
\frac{q}{4\pi\varepsilon_0 r}
\biggl(
1-\frac12\frac{-2az+a^2}{r^2} + \frac38 \biggl(\frac{-2az+a^2}{r^2}\biggr)^2+\dots
\biggr)
\\
&\qquad
-
\frac{q}{4\pi\varepsilon_0 r}
\biggl(
1-\frac12\frac{2az+a^2}{r^2} + \frac38 \biggl(\frac{2az+a^2}{r^2}\biggr)^2+\dots
\biggr)
\\
&=
-\frac{2aq}{4\pi\varepsilon_0r^3}z + \dots
\end{align*}
Wenn die beiden Ladungen näher zusammen rücken, wenn also $a\to 0$,
dann verschwindet das Potential.
Wenn wir das Potential weiterhin sehen wollen, müssen wir den Betrag
der Ladungen entsprechend vergrössern.
Wenn wir $a$ gegen $0$ gehen lassen, lassen wir gleichzeit die Ladung
$q$ grösser werden, so dass das Produkt $d=2qa$ gleich bleibt.
Mit dieser Konvention wird das Dipolpotential
\begin{equation}
f(x,z) = -\frac{d}{4\pi\varepsilon_0 r^2}\frac{z}{r}+\dots
\label{skript:multipol:dipolpotential}
\end{equation}
Darin haben wir statt $z$ den Bruch $z/r$ abgespalten, da dieser
eintlang eines vom Nullpunkt ausgehenden Strahls vom Zentrum jeweils
konstant ist.
Wir haben daher das Potential in Faktoren aufgeteilt, die verschiedene
geometrische Bedeutung haben.
Der Faktor $1/r^2$ beschreibt, wie das Potential mit der Entfernung abnimmt.
Der Faktor $z/r$ beschreibt, wie das Potential von der Richtung im
Bezug auf die $z$-Achse abhängt.
Die übrigen Faktoren beschreiben, wie das Potential aus dem Dipolmoment
$d$ erzeugt wird.

\subsubsection{Vektorschreibweise}
Das Dipolpotential kann besonders elegant geschrieben werden, wenn
wir das skalare Dipolmoment $d$ durch einen Vektor $\vec{d}$ ersetzen.
In der Formel~\eqref{skript:multipol:dipolpotential}
für das Dipolpotential brauchen wir die $z$-Koordinate
des Punktes.
Diese können wir als das Skalarprodukt mit dem Standardbasisvektor
in $z$-Richtung bekomen.
Wenn wir also
\[
\vec{d}=\begin{pmatrix}0\\0\\d\end{pmatrix}
\]
setzen, dann können wir das Dipolpotential vektoriell als
\begin{equation*}
f(\vec{r})
=
-
\frac{1}{4\pi\varepsilon_0r^2} \frac{\vec{d}\cdot\vec{r}}{r}
\end{equation*}
geschrieben werden.
In dieser Form ist das Dipolpotential für jede beliebige Orientierung
des Dipolmomentes verwendbar.

Der Dipolterm geht für $r\to\infty$ wie $r^{-2}$ gegen $0$, also
deutlich schneller als das Potential einer Punktladung.

\subsection{Quadrupol}
Wie in der eindimensionalen Situation vermuten wir, dass sich
das Potential komplizierterer Ladungsverteilungen ebenfalls durch
eine Reihe darstellen lässt, deren nächster Term von der Form
\begin{equation}
\frac1{4\pi\varepsilon_0}\cdot\frac{1}{r^5} p(\vec{r},\vec{r})
\label{skript:multipol:quadropolterm}
\end{equation}
sein muss.
Darin ist $p$ ein Ausdruck, der in beiden Argumenten linear ist.

Schreiben wir die Koordinaten als $(x_1,x_2,x_3)$, dann wird 
sich $p$ in der Form
\[
\sum_{k,l=1}^3 Q_{kl}x_kx_l
\]
schreiben lassen.
Allerdings kann nicht jede beliebige Matrix zugelassen werden, 
da ja nur die Abweichungen vom Dipolmoment erfasst werden sollen.
Nimmt man für $Q$ die Einheitsmatrix, dann erhält man einfach nur
\[
\sum_{k,l=1}^3 Q_{kl}x_kx_l=\sum_{i=1}^3 x_i^2 = \vec{r}\cdot\vec{r}=r^2,
\]
der Quadrupol-Term wird also 
\[
\frac1{4\pi\varepsilon_0}\cdot\frac{1}{r^3},
\]
was keine Richtungsabhängigkeit mehr enthält und wir daher erwarten
würden, dass diese Art von Abhängigkeit bereits im ersten Term 
enthalten war.
Man kann dies zum Beispiel dadurch erreichen, dass die Matrix $Q$ 
verschwindende Spur haben soll, also $\operatorname{Spur}Q=0$.

Die defaillierte Berechnung des Quadrupolanteils ist etwas mühsam,
wir geben hier nur das Resultat an.
Man findet
\[
Q_{kl}
=
\int_{\mathbb R^3}
\varrho(\vec{r}) \cdot (3x_kx_l-r^2\delta_{kl})
\,dx_1\,dx_2\,dx_3.
\]
Das Wachstum des Quadrupolterms~\eqref{skript:multipol:quadropolterm}
ist also von der Ordnung $r^{-3}$.
Auch dieser Term geht wieder einer Potenz schneller gegen $0$ als der
Dipolterm.

\subsection{Höhere Multipole}
Die Entwicklung Entwicklung in Dipol und Quadrupol lässt sich noch
weiter führen.
Dabei werden sukzessive Terme entstehen, die immer schneller gegen $0$
gehen, für grosse Entfernung vom Nullpunkt also immer weniger von
Bedeutung sein werden.
Die Darstellung dieser höheren Multipole wird allerdings zunehmen
schwierig. 
Der nächste Term nach dem Quadrupolterm müsste die Form
\begin{equation}
\frac1{4\pi\varepsilon_0}\frac{p(\vec r)}{r^7}
\label{skript:multipol:7ord}
\end{equation}
haben, wobei der Zähler $p(\vec r)$ eine Grösse dritter Ordnung in
den Koordinaten sein müsste.
Er wird also durch ein homogenes Polynom dritten Grades in den Koordinaten
$x$, $y$ und $z$ beschrieben.
Der Term~\eqref{skript:multipol:7ord} geht wie $r^{-4}$ gegen Null,
also erneut eine Potenz schneller als der Quadrupolterm.

\section{Zusammenfassung}
Allen Termen der Multipolentwicklung
\[
f(\vec r)
=
\frac{1}{4\pi\varepsilon 0}
\biggl(
q
\frac{1}{r}
+
\frac{1}{r^2} \frac{\vec{d}\cdot\vec{r}}{r}
+
\frac{1}{r^3} \sum_{k,l}Q_{kl}\frac{x_k}{r}\frac{x_l}{r}
+
\frac{1}{r^4} \sum_{k,l,j}Q_{klj}\frac{x_k}{r}\frac{x_l}{r}\frac{x_j}{r}
+
\dots
\biggr)
\]
ist gemeinsam, dass sie die Abhängigkeit des Potentials in einen
aus physikalischen Gründen plausiblen radialen Teil der Form $r^{-k}$
und einen richtungsabhängigen Teil aufteilen.
Die reine Richtungsabhängigkeit kann dadurch ausgedrückt werden, dass
er nur von den Quotienten $x/r$, $y/r$ und $z/r$ abhängt.

TODO:
\begin{itemize}
\item Potentiallinien-Plots für Dipol- und Quadrupolfeld
\item 3D-Plot von Dipolmoment und evtl Quadropolfeld
\end{itemize}


