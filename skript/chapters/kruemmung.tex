%
% kruemmung.tex
%
% (c) 2017 Prof Dr Andreas Müller, Hochschule Rapperswil
%
\chapter{Krümmung\label{skript:chapter:kruemmung}}
\lhead{Krümmung}
\rhead{}
Das Konzept der Krümmung einer Kurve wird bereits im ersten Semester
in der Analysis eingeführt.
Die Kurve kann zum Beispiel durch die Länge entlang der Kurve
parametrisiert werden.
Durch Festlegung eines Nullpunkts der Längenmessung erhält die
Kurve damit ein Koordinatensystem, welches die Punkte auf der
Kurve eindeutig identifizieren lässt.
Es stellt sich heraus, dass Krümmung eine Eigenschaft der Einbettung
einer Kurve in einen höherdimensionalen Raum ist.
Durch Messungen allein innerhalb der Kurve, also durch Verwendung
des eben beschriebenen Koordinatensystems lassen sich verschiedene
gekrümmte Kurven nicht unterscheiden.

Dies ändert in höheren Dimension.
Eine Fläche kann zwar immer noch auf ganz verschiedene Art in
einen dreidimensionalen Raum eingebettet werden.
Ein Stück Papier kann man sich zum Beispiel als Ausschnitt einer
Ebene vorstellen, man kann es aber auch zu einem Zylinder oder
Kegel zusammenrollen.
Durch Messungen allein innerhalb der Fläche, sind diese verschiedenen
Einbettungen nicht unterscheidbar.
Eine Ameise auf der Fläche könnte keinen Unterschied feststellen.
Eine Kugeloberfläche dagegen liesse sich sehr wohl durch Messungen
allein innerhalb der Fläche unterscheiden.
Misst man zum Beispiel von einem Kreis um einen Punkt den Umfang,
dann stellt man fest, dass der auf einer Ebene der Umfang immer $2\pi r$ ist.
Auf einer Kugeloberfläche dagegen ist der Umfang kleiner.
Der Radius wird ja nicht in gerader Linie in einer Eben gemessen,
sondern entlang einer gekrümmten Linie, die sich von der Tangentialebene
entfernt.

Die Physik seit Galileo und Newton machte die Annahme, dass die
Geometrie des Raumes durch ein dreidimensionales rechtwinkliges
Koordinatensystem mit der Längenmessungsformel
\begin{equation}
l=\sqrt{\Delta x^2+\Delta y^2+\Delta z^2}
\label{skript:kruemmung:pytagoras}
\end{equation}
adäquat beschrieben wird.
Diese Annahme entsprach zwar der Erfahrung, doch es gab keine
Begründung dafür.
Der Philosoph Emanuel Kant konnte sich zwar keine andere Geometrie
vorstellen, doch mit den Erkenntnissen von Bolyai, Lobaschevski
und Gauss wurde klar, dass es durchaus denkbare andere Geometrien
gibt.
Bernhard Riemann hat dann auch Methoden entwickelt, wie man die
Geometrie studieren kann, indem man ausschliesslich die Längenmessung
innerhalb des Raumes verwendet.
Damit ist die Geometrie unseres Raumes nicht länger einfach das
Resultat einer axiomatischen Beschreibung, wie Euklid sie gegeben hat,
vielmehr ist sie zu einer experimentellen Wissenschaft geworden.

In diesem Kapitel wird daher untersucht, wie die Geometrie des Raumes
mit der Längenmessung zusammenhängt und wie Krümmung charakterisiert
werden kann.
Dazu wird der Begriff der Geodäten verwendet, der auch zum Beispiel
in der Vermessung eine Rolle spielt.
Auch die Ausbreitung des Lichts in einem Medium ist einer solchen
Beschreibung zugänglich.
Das Licht wählt immer den Weg mit der geringsten Laufzeit, nicht
unbedingt den geometrisch kürzesten Weg.
Daher können sich Lichtstrahlen in einem inhomogenen Medium
krümmen.
Aus der Perspektive des Lichtes ist aber nicht die Bahn gekrümmt,
es folgt der in dieser Geometrie geradest möglichen Bahn.
Nicht die Bahn ist gekrümmt, sondern die Längenmessung weicht von
der üblichen \eqref{skript:kruemmung:pytagoras} ab, der Raum ist
gekrümmt.

Schon aus diesen Beispielen wird klar, dass man sich einen gekrümmten
Raum nicht unbedingt als etwas vorstellen muss, was ``krumm'' in einen
grösseren Raum eingebettet ist. 
Es ist zwar so, das auf diese Weise gekrümmte Räume entstehen können,
aber das allgemeine Verständnis sagt einfach nur, dass die Längenmessung
nicht so funktioniert wie \eqref{skript:kruemmung:pytagoras} suggeriert.

Wir werden wie folgt vorgehen.
Im ersten Abschnitt wird die Krümmung in einer Dimension definiert,
es geht hier vor allem darum daran zu erinnern, wie die zweiten
Ableitungen eingehen.
Im zweiten Abschnitt wird der Begriff der Längenmessung auf einer
Fläche und allgemein in einem höherdimensionalen Raum beschrieben
und an einigen Bespielen studiert.
Der kürzeste Weg zwischen zwei Punkten heisst Geodäte, die Gleichung
einer Geodäten wird im dritten Abschnitt hergeleitet.
Dazu entwickeln wir einen umfangreichen Formalismus, der später auch
zur Berechnung der Krümmung nützlich sein wird.
Im vierten Abschnitt untersuchen wir, wie der Versuch, einen Tangentialvektor
entlang einer Kurve zu transportieren, in gewissen Fällen vom 
Weg abhängt.
Wir werden dies als ein Anzeichen von Krümmung werten.
Es folgt die Definition einer Grösse, mit der man die Krümmung messen
kann.

In den folgenden Abschnitten geben wir einen Ausblick darauf, wie 
die moderne Physik unseren Raum als einen gekrümmten Raum beschreibt.
An einfachen Modellen soll gezeigt werden, wie man sich die Gravitationswirkung
als die Wirkung eines gekrümmten Raumes vorstellen kann, wie schwarze Löcher
beschrieben werden können, und wie alle diese Dinge tatsächlich gemessen
werden können.

%
% k-id.tex -- Krümmung eines eindimensionalen Raumes, Einbettung
%
% (c) 2017 Prof Dr Andreas Müller, Hochschule Rapperswil
%
\section{Krümmung einer Kurve
\label{skript:kruemmung:section:kurve}}
\rhead{Krümmung einer Kurve}
Wir betrachten eine Kurve in der Ebene durch einen Parameterdarstellung
$c\colon [a,b] \to\mathbb R^2:t\mapsto c(t)$.
Man kann die Parametrisierung als die Wahl eines Koordinatensystems
für die Kurve interpretieren.
Es stellt sich daher die Frage, inwieweit Aussagen über die Kurve von
der Wahl des Koordinatensystems abhängig sind.

In $\mathbb R^2$ haben wir eine Längenmessung zur Verfügung, welche
wir dazu verwenden können, die Länge der Kurve zu berechnen.
Die Funktion $s(t)$ soll die Länge der Kurve zwischen den Parameterwerten
$a$ und $t$ berechnen, wir wissen also insbesondere, dass $s(0)=0$.
Damit das Konzept der Länge der Kurve überhaupt sinnvoll ist, müssen
wir verlangen, dass die Kurve ausreichend glatt ist.
Wir verlangen daher, dass die Funktion $c(t)$ einmal stetig differenzierbar
ist.
Wir bezeichnen die Ableitung mit einem Punkt.

\subsection{Bogenlängenparameter}
Interpretieren wir den Parameter $t$ als die Zeit der Bewegung eines 
Teilchens entlang der Kurve, dann ist die zeitliche Änderung der Strecke,
die das Teilchen zurück gelegt hat, gerade die Länge des
Geschwindigkeitsvektors, also
\[
\frac{d}{dt} s(t) = \biggl|\frac{d}{dt}c(t)\biggr| = |\dot c(t)|.
\]
Die Funktion $s(t)$ ist also die Lösung der Differentialgleichung
\begin{equation}
\dot s(t)=|\dot c(t)|
\quad
\text{ mit der Anfangsbedingung }\quad s(a)=0.
\label{skript:kruemmung:laengedgl}
\end{equation}
Da $|\dot c(t)|$ eine stetige Funktion ist können wir nach allgemeinen
Sätzen der Theorie der gewöhnlichen Differentialgleichungen davon ausgehen,
dass diese Differentialgleichung eine eindeutig bestimmte Lösung hat.

Die Differentialgleichung \eqref{skript:kruemmung:laengedgl} kann 
durch Integrieren sofort gefunden werden, sie ist
\begin{equation}
s(t)
=
\int_a^t \dot s(\tau)\,d\tau
=
\int_a^t |\dot c(\tau)| \,d\tau
=
\int_a^t \sqrt{\dot x(\tau)^2 + \dot y(\tau)^2}\,d\tau,
\label{skript:kruemmung:laenge}
\end{equation}
wie man auch durch Ableiten nach $t$ sofort nachprüfen kann.

Verlangen wir von der Kurvenparametrisierung $c(t)$ zusätzlich,
dass $|\dot c(t)|\ne 0$ für alle Parameterwerte $t$, dann ist
$s(t)$ sogar eine streng monoton wachsende Funktion, insbesondere
kann sie invertiert werden.
Wir können daher zu jedem gegebenen Wert der Länge $s$ durch
Anwendung der inversen Funktion sofort einen Parameterwert $t$
finden, bei dem der Wert $s$ der Länge erreicht wird, also
$s(t)=s$.
Wir schreiben diesen Wert auch abkürzen als $t(s)$, dies ist eine
vereinfachte Notation für die inverse Funktion von $s(t)$.
Die Ableitung der Funktion $s(t)$ ist
\begin{equation}
\frac{d}{ds} t(s) = \frac{1}{\dot s(t(s))} = \frac1{|\dot c(t(s))|}
\label{skript:kruemmung:abllaengeinv}
\end{equation}
nach der Formel für die Ableitung der inversen Funktion.

Zu einer beliebigen differenzierbaren Kurve mit der Eigenschaft
$|\dot c(t)|\ne 0$ können wir also immer eine neue Parametrisierung
$c(s)=c(t(s))$ finden, welche die Eigenschaft hat, dass $s$ die
Länge entlang der Kurve gerade die Kurvenlänge ist, die durchlaufen
wird.
Wir nennen $s$ auch den {\em Kurvenlängenparameter} entlang der
Kurve.
\index{Kurvenlängenparameter}
Wir sagen auch, $c(s)$ ist durch Kurvenlänge parametrisiert.

Die neue Parametrisierung hat als ``Geschwindigkeitsvektor''
\[
\frac{d}{ds} c(s)
=
\frac{d}{ds} c(t(s))
=
\dot c(t(s)) \frac{d}{ds}t(s)
\]
seine Länge wird unter Verwendung von
\eqref{skript:kruemmung:abllaengeinv}
\[
\biggl|
\frac{d}{ds} c(s)
\biggr|
=
\biggl|\dot c(t(s))\frac1{|\dot c(t(s))|}\biggr|
=
1,
\]
bei Verwendung eines Kurvenlängenparameters wird der Tangentialvektor
immer ein Einheitsvektor sein.

Umgekehrt wird die Differentialgleichung \eqref{skript:kruemmung:laengedgl}
besonderes einfach, wenn $|\dot c(t)|=1$ ist, nämlich $\dot s(t)=1$ mit
der Lösung $s(t)=t$.
Die Bedingung $|\dot c(t)|=1$ charaktersiert also $t$ als einen
Kurvenlängenparameter.

\begin{beispiel}
Wir betrachten eine Gerade 
\[
c(t)=\vec p + t\vec v,\qquad \vec v\ne 0
\]
in der Ebene.
Wir legen $a=0$ willkürlich fest.
Die Differentialgleichung~\eqref{skript:kruemmung:laengedgl} für die
Kurven\-längen\-funktion $s(t)$ wird
\[
\dot s(t) = |\dot c(t)| = |v|
\]
mit der Lösung
\[
s(t)=|v|t.
\]
Da $v\ne 0$ ist $s(t)$ monoton wachsend, und wir können sofort nach
$t$ auflösen, es ist
\[
t(s)=\frac{s}{|\vec v|}.
\]
Setzen wir dies in die ursprüngliche Parametrisierung ein, erhalten
wir
\[
c(s)
=
c(s(t))
=
\vec p + t(s)\vec v
=
\vec p + \frac{s}{|\vec v|}\vec v
=
\vec p + s\frac{v}{|v|},
\]
Die Ableitung nach $s$ hiervon ist ein Einheitsvektor, wie zu erwarten
war.
\end{beispiel}

\begin{beispiel}
Wir parametrisieren den Einheitskreis mit Hilfe der $x$-Koordinate, also
durch
\[
c\colon
[-1,1]\to\mathbb R^2
\colon
t\mapsto (-t,\sqrt{1-t^2}).
\]
Ganz offensichtlich ist dies kein Kurvenlängenparameter.
Für die Kurvenlänge brauchen wir die Ableitung nach dem Parameter $t$,
sie ist
\[
\dot c(t)=\biggl(-1, -\frac{t}{\sqrt{1-t^2}}\biggr).
\]
Die Kurvenlänge ist das Integral der Länge der Ableitung:
\[
s(t)
=
\int_{-1}^t \sqrt{1 + \frac{\tau^2}{1-\tau^2}}\,d\tau
=
\int_{-1}^t \sqrt{\frac{1-\tau^2 +\tau^2}{1-\tau^2}}\,d\tau
=
\int_{-1}^t \frac{1}{\sqrt{1-\tau^2}}\,d\tau
\]
Das Integral auf der rechten Seite kann man aber in geschlossener
Form auswerten, es ist
\[
s(t)=\arcsin(t) + \frac{\pi}{2}
\]
mit der inversen Funktion
\[
t(s)
=
\sin\biggl(s-\frac{\pi}2\biggr)
=
-\cos(s)
\]
Setzen wir dies in die Parametrisierung ein, erhalten wir
\[
c(s)
=
\bigl(\cos(s), \sqrt{1-\cos^2(s)}\bigr)
=
(\cos(s), \sin(s)),
\]
also die übliche Parametrisierung des Einheitskreises mit Hilfe
des Winkels.
Der Winkel muss in Bogenmass gemessen werden, also der Länge entlang
des Kreises.
\end{beispiel}

Da sich jede beliebige Kurve mit einem Bogenlängenparameter versehen
lässt, können wir für je zwei beliebige Kurven immer eine Abbildung
zwischen den Kurven definieren, die Punkte mit gleichem Bogenlängenparameter
zur Deckung bringt.
Zwei Kurven sind also nicht mehr unterscheidbar, wenn sie mal mit
einem Bogenlängenparameter ausgestattet sind.
Verbleibende Unterschiede zwischen Kurven entstehen einzig durch die
Einbettung der Kurve im Raum.

Für eine Ameise, die ausschliesslich auf der Kurve lebt, ist die Länge,
die sie entlang der Kurve zurückgelegt hat, die einzige geometrische
Information, die ihr zugänglich ist.
Sie kann verschiedene Kurven im Raum nicht voneinander unterscheiden.

\subsection{Krümmung}
Wir möchten in diesem Abschnitt zeigen, dass die Krümmung einer Kurve
ausschliesslich eine Eigenschaft der Einbettung der Kurve in den
Raum ist.
Wir tun das, indem wir davon ausgehen, dass eine Kurve bereits mit
der Kurvenlänge parametrisiert ist, dass wir also die Standardparametrisierung
gewählt haben, in der die über die innere Geometrie der Kurve verfügbare
Information vereits vollständig verwertet worden ist.

Eine Gerade zeichnet sich dadurch aus, dass der Tangentialvektor in
der Bogenlängenparametrisierung nicht ändert.
Eine Gerade ist nicht gekrümmt.
Krümmung ist also die Änderung des Tangentialvektors oder
\[
\kappa(s)=|\ddot c(s)|.
\]
\index{Krümmung einer Kurve}

\begin{beispiel}
Der Kreis mit Radius $r$ kann parametrisiert werden durch die Funktion
\[
s\mapsto c(s)=\biggl( r\cos\frac{s}{r}, r\sin\frac{s}{r}\biggr).
\]
Die Ableitung ist
\[
\dot c(s)=\biggl(\cos\frac{s}{r},\sin\frac{s}{r}\biggr)
\qquad\Rightarrow\qquad
|\dot c(s)|
=
\sqrt{\cos^2\frac{s}{r}+\sin^2\frac{s}{r}}=1,
\]
dies ist also eine Parametrisierung mit Bogenlängenparameter.
Die Krümmung ist der Betrag der zweiten Ableitung, also
\[
\ddot c(s)=\biggl(\frac1r\cos\frac{s}{r},\frac1r\sin\frac{s}{r}\biggr)
\qquad\Rightarrow\qquad
\kappa(s)
=
|\ddot c(s)|
=
\frac1r.
\]
Der Kreis mit Radius $r$ ist also eine Kurve mit konstanter Krümmung
$\kappa(s)=\frac1r$.
\end{beispiel}

Ist die Kurve nicht durch die Bogenlänge parametrisiert, wird die
Berechnung der Krümmung etwas komplizierter.
Es muss aber immer noch gelten, dass $\kappa(t)$ die Änderung des
Einheitstagentialvektors pro Längeneinheit entlang der Kurve angibt.
Es ist also zu berechnen
\begin{equation}
\kappa(t)
=
\underbrace{\frac{1}{|\dot c(t)|}}_{\text{pro Längeneinheit}}
\cdot\;
\biggl|\frac{d}{dt}\underbrace{\frac{\dot c(t)}{|\dot c(t)|}}_{\text{Einheitstangentialvektor}}\biggr|.
\label{skript:kruemmung:krallg2}
\end{equation}
Die Berechnung ist etwas mühsam wegen des Vektorbetrages im Nenner.
Wir berechnen daher zuerst die Ableitung von $|\dot c(t)|$ mit
Hilfe der Kettenregel:
\begin{align*}
\frac{d}{dt}|\dot c(t)|
&=
\frac{d}{dt}\sqrt{\dot x(t)^2+\dot y(t)^2}
=
\frac1{2\sqrt{\dot x(t)^2 + \dot y(t)^2}}(2\dot x(t)\ddot x(t) + 2\dot y(t)\ddot y(t))
\\
&=
\frac{1}{|\dot c(t)|}
\begin{pmatrix}\dot x(t)\\\dot y(t)\end{pmatrix}
\cdot
\begin{pmatrix}\ddot x(t)\\\ddot y(t)\end{pmatrix}
=
\frac{1}{|\dot c(t)|}
\dot c(t)\cdot\ddot c(t)
=
\frac{\dot c(t)}{|\dot c(t)|} \cdot\ddot c(t).
\end{align*}
Damit können wir jetzt auch die Ableitung des Einheitstangentialvektors
berechnen:
\begin{align}
\frac{d}{dt} \frac{\dot c(t)}{|\dot c(t)|}
&=
\frac{\ddot c(t) |\dot c(t)| - \dot c(t) \frac{d}{dt}|\dot c(t)|}{|\dot c(t)|^2}
=
\frac{\ddot c(t) |\dot c(t)| - \dot c(t) \frac{\dot c(t)}{|\dot c(t)|}\cdot \ddot c(t)}{|\dot c(t)|^2}
=
\frac{\ddot c(t) - \frac{\dot c(t)}{|\dot c(t)|} \left( \frac{\dot c(t)}{|\dot c(t)|}\cdot \ddot c(t)\right)}{|\dot c(t)|}
\label{skript:kruemmung:krallg1}
\end{align}
Schreiben wir den Einheitstangentialvektor
\[
\vec e(t)= \frac{\dot c(t)}{|\dot c(t)|},
\]
wird der Zähler etwas übersichtlicher zu
\[
\ddot c(t) - \frac{\dot c(t)}{|\dot c(t)|} \biggl( \frac{\dot c(t)}{|\dot c(t)|}\cdot \ddot c(t)\biggr)
=
\ddot c(t) - \vec e(t)\; \bigl(\vec{e}(t)\cdot \ddot c(t)\bigr).
\]
Da $\vec e(t)$ ein Einheitsvektor ist, ist
$\vec e(t) \;(\vec e(t)\cdot\ddot c(t))$ die Projektion des Vektors
$\ddot c(t)$ auf den Vektor $\vec c(t)$.
Der Nenner ist also die Komponenten von $\ddot c(t)$ senkrecht auf dem
Tangentialvektor.
Wir brauchen davon aber nur den Betrag, also die Höhe des Parallelograms
mit den Seiten $\vec e(t)$ und $\ddot c(t)$.
Diese können wir aber auch dadurch erhalten, dass wir den orientierten
Parallelogramminhalt berechnen, und durch die Länge der Grundseite
$\vec e(t)$ teilen, die aber ein Einheitsvektor ist.
Den Parallelogramminhalt können wir mit der Determinante berechnen:
\[
\biggl|\ddot c(t) - \frac{\dot c(t)}{|\dot c(t)|}
\biggl(\frac{\dot c(t)}{|\dot c(t)|} \cdot \ddot c(t))\biggr)\biggr|
=
\det \biggl(\frac{\dot c(t)}{|\dot c(t)|}, \ddot c(t))\biggr)
=
\frac1{|\dot c(t)|}\det \dot c(t),\ddot c(t)).
\]
Daraus können wir jetzt auch eine Formel für die Krümmung zusammensetzen.
Zunächst müssen wir den Faktor $|\dot c(t)|$ im Nenner wieder
hinzufügen, den wir in~\eqref{skript:kruemmung:krallg1} gefunden hatten.
Ausserdem brauchen wir einen weiteren Faktor $|\dot c(t)|$ im Nenner
aus der Definition~\eqref{skript:kruemmung:krallg2}.
Damit wird die allgemeine Formel
\begin{equation}
\kappa(t)
=
\frac{\det(\dot c(t),\ddot c(t))}{|\dot c(t)|^3}
\label{skript:kruemmung:krkurveallg}
\end{equation}
für die Krümmung einer ebenen Kurve.


\begin{beispiel}
Wir berechnen die Krümmung einer Ellipse mit den Halbachsen
$a$ und $b$, und verwenden dazu die Parametrisierung mit Hilfe des
Zentriwinkels, also
\[
c
\colon
[0,2\pi]\to\mathbb R^2
\colon
t\mapsto c(t) = (a\cos t, b \sin t).
\]
Nach der Formel~\eqref{skript:kruemmung:krkurveallg} brauchen wir zunächst
die ersten beiden Ableitungen, diese sind
\begin{align*}
\dot c(t)
&=
(-a\sin t, b\cos t)
\\
\ddot c(t)
&=
(-a\cos t, -b\sin t)
\end{align*}
Die Determinante von $\dot c(t)$ und $\ddot c(t))$ ist
\[
\det(\dot c(t), \ddot c(t))
=
\left|\begin{matrix}
-a\sin t & -a \cos t\\
 b\cos t & -b \sin t
\end{matrix}\right|
=
ab\sin^2t+ab\cos^2 t
=
ab.
\]
Die Krümmung ist daher
\[
\kappa(t)
=
\frac{ab}{(a^2\sin^2 t + b^2 \cos^2 t)^{\frac32}}
\]
Da die Faktoren $\sin^2t$ und $\cos^2t$ im Nenner sich zu $1$ summieren,
steht in der Klammer im Nenner ein gewichteter Mittelwert von $a^2$
und $b^2$.
Die Extremwerte des Nenners sind daher $a^3$ und $b^3$, sie werden bei
$t=\pm\frac{\pi}2$ bzw.~$t\in\{0,\pi\}$ angenommen.
Nimmt man an, dass $a>b$ ist, dann wird die maximale Krümmung $t=0$ und
und $t=\pi$ erreicht, sie ist $ab/b^3=a/b^2$.
Die minimale Krümmung wird dagegen bei $t=\pm\frac{\pi}2$ angenommen,
sie ist $ab/a^3=b/a^2$.

Ein Kreis ist eine Ellipse, bei der $a$ und $b$ übereinstimmen, dann sind
die beiden Extremwerte gleich gross, nämlich
\[
\frac{a}{b^2}=\frac{r}{r^2}=\frac1r
\qquad\text{und}\qquad
\frac{b}{a^2}=\frac{r}{r^2}=\frac1r,
\]
wie wir früher bereits gefunden haben.
\end{beispiel}

Für eine Kurve in drei Dimension lässt sich das bei der Herleitung
der Formel~\eqref{skript:kruemmung:krkurveallg} für $\kappa(t)$
verwendete Argument fast unverändert übertragen.
Die einzige Änderung ist an der Stelle erforderlich, wo wir die Determinante
zur Berechnung des orientierten Volumens des Parallelograms verwendet
haben.
In drei Dimensionen muss stattdessen der Betrag des Vektorproduktes 
verwendet werden.
Die Krümmung einer Raumkurve in einer beliebigen Parametrisierung ist
daher
\begin{equation}
\kappa(t)
=
\frac{|\dot c(t)\times \ddot c(t)|}{|\dot c(t)|^3}.
\label{skript:kruemmung:krkurveallg3d}
\end{equation}
Dies ist natürlich identisch mit \eqref{skript:kruemmung:krkurveallg},
wenn man sich die Ebene der Kurve in einen dreidimensionalen Raum
eingebettet vorstellt.

\begin{beispiel}
Wir betrachten die beiden Kurven
\begin{align*}
c_1(t)&=(t,t^2,0)
&
c_2(t)&=(t,t^2,t^3).
\end{align*}
$c_1(t)$ ist eine ebene Kurve, nämlich die Projekten der Raumkurve
$c_2(t)$ in die $x$-$y$-Ebene.
Wir wollen von beiden Kurven die Krümmung berechnen.
Dazu berechnen wir zunächst die Ableitungen und das Vektorprodukt
\begin{align*}
\dot c_1(t)
&=
(1,2t,0)
&
\dot c_2(t)
&=
(1,2t,3t^2)
\\
\ddot c_1(t)
&=
(0,2,0)
&
\ddot c_2(t)
&=
(0, 2, 6t)
\\
\dot c_1(t)\times \ddot c_1(t)
&=
(0,0,2)
&
\dot c_2(t)\times \ddot c_2(t)
&=
(6t^2,-6t,2)
\end{align*}
Daraus kann man jetzt die Krümmungen berechnen:
\begin{align*}
\kappa_1(t)
&=
\frac{2}{(1+4t^2)^{\frac32}}
\\
\kappa_2(t)
&=
\frac{2}{(1+4t^2 + 9t^4)^{\frac32}}
\sqrt{1+9t^2+9t^4}
\end{align*}
Für den Parameterwert $t=0$ stimmen die beiden Krümmungen überein.
\end{beispiel}

\subsection{Rekonstruktion einer ebenen Kurve aus der Krümmung}
Die Krümmung bestimmt eine Kurve eindeutig.
Umd dies einzusehen, parametrisieren wir eine ebene Kurve mit
$x$, schreiben also $c(x)=(x,y(x))$.
Wir müssen jetzt nachrechnen, dass die Vorgabe der Krümmung $\kappa(x)$
und der Steigung in einem Punkt die Kurve eindeutig festlegt.
Dazu stellen wir eine Differentialgleichung zweiter Ordnung auf,
allgemeine Sätze über Existenz und Eindeutigkeit der Lösung von
gewöhnlichen Differentialgleichungen werden unsere Aussage dann
als Konsequenz haben.

In dieser speziellen Wahl der Parametrisierung ist $\dot x = 1$ und
$\ddot x=0$.
Die Ableitung von $y$ nach dem Parameter ist dann $\dot y=y'(x)$.
Somit wird die Formel für die Krümmung
\[
\kappa(x)
=
\frac{\left|\begin{matrix}\dot x&\ddot x\\\dot y&\ddot y\end{matrix}\right|}{(\dot x^2+\dot y^2)^{\frac32}}
=
\frac{\left|\begin{matrix}1&0\\y'&y''\end{matrix}\right|}{(1+y'^2)^{\frac32}}
=
\frac{y''}{(1+y'(x))^{\frac32}}
\]
der in expliziter Form:
\[
y''(x)=\kappa(x) (1+y'(x)^2)^{\frac32}.
\]
Dies ist eine gewöhnliche Differentialgleichung zweiter Ordnung.
Da die rechte Seite erfüllt die Bedingungen, die in üblichen
Eindeutigkeitssätzen für gewöhnliche Differentialgleichungen
verlangt werden, daher gibt es genau eine Lösung dieser
Differentialgleichung zu vorgegebenem Anfangspunkt und Anfangsrichtung.
Damit ist gezeigt, dass in der Ebenen zwei Kurven mit der gleichen
Krümmung, gleichem Anfangspunkt und gleicher Anfangsrichtung übereinstimmen
müssen.
Anders formuliert: zwei ebene Kurven mit gleicher Krümmung in Abhängigkeit
von einem Bogenlängenparameter sind kongruent.

In drei Dimensionen kann die Krümmung allein eine Raumkurve nicht bestimmen.
Man kann dies zum Beispiel so einsehen.
Ist $c(t)$ eine gegeben Raumkurve, dann können wir deren Krümmung
$\kappa(t)$ berechnen.
In jeder Ebene durch den Punkt $c(0)$, die auch den Tangentenvektor
$\dot c(t)$ enthält, gibt es eine ebenen Kurve mit der gleichen
Krümmung $\kappa(t)$.
Da es unendlich viele solche Ebenen gibt, gibt es unendlich viele 
ebene Kurven, die die gleiche Krümmung haben wie die vorgegebene
Raumkurve.

Um eine Raumkurve eindeutig festzulegen braucht es daher ein weiteres
Datum, welches beschreibt, wie sich die aktuelle Tangentialebene
aufgespannt von $\dot c(t)$ und $\ddot c(t)$ dreht.
Diese {\em Torsion} genannt Grösse bestimmt dann die Kurve vollständig.
\index{Torsion}
Für eine ebene Kurve verschwindget die Torsion.

Die Kurventheorie lässt sich sogar auf eine beliebig grosse Zahl
von Dimensionen verallgemeinern.
Es stellt sich heraus, dass mit jeder zusätzlichen Dimension eine
zusätzliche Funktion Funktion notwendig wird.
Für eine Kurve in $n$ Dimensionen braucht es also $n-1$ Funktionen,
die beschreiben, wie sich ein entlang der Kurve mitbewegtes Koordinatensystem,
das sogenannte Frenet-$n$-Bein ändert.
\index{Frenet-$n$-Bein}
Diese Funktionen legen dann die Kurve bis auf die Wahl einer Anfangslage
des Koordinatensystem eindeutig fest.
Mit anderen Worten, wenn zwei Kurven in $n$ Dimensionen in den genannten
$n-1$ Funtionen übereinstimmen, dann sind sie kongruent.



%
% k-laenge.tex -- Längenmessung in einer Fläche oder in einem Raum
%
% (c) 2017 Prof Dr Andreas Müller, Hochschule Rapperswil
%
\section{Längenmessung
\label{skript:kruemmung:section:laengenmessung}}
\rhead{Längenmessung}
Da wir nicht weiter annehmen wollen, dass sich der Raum mit Hilfe
eines rechtwinkligen Koordinatensystems adäquat beschreiben lässt,
müssen wir automatisch beliebige Koordinatensysteme zulassen.
Je nach Wahl eines Koordinatensystems werden dann Vektoren, die
wir für die Beschreibung der physikalischen Gesetze benötigen,
verschiedenen Koordinaten haben.
Da alle diese Koordinatensystem gleichberechtigt sind, müssen
wir Vektoren auf einheitliche Art umrechnen können.
Ausserdem müssen die Naturgesetzt so formuliert sein, dass
sie in jedem beliebigen Koordinatensystem gleich aussehen.

In diesem Abschnitt beginnen wir damit, den Begriff der Längenmessung
auf beliebige Koordinatensysteme auszudehnen.
Ein Punkt wird beschrieben durch seine Koordinaten, die wir mit
$x^\mu$ bezeichnen, wobei $\mu$ von $1$ bis $n$ läuft, $n$
ist die Dimension. 
Die etwas ungewohnte Schreibweise für die Indizes wie Exponenten
hat einen tieferen Grund in der Tensorrechnung, und wird später
verständlich werden.

\subsection{Vektoren und Koordinatentransformation}
Eine Koordinatentransformation zwischen zwei Koordinatensystemen ist
eine Abbildung, die die Koordinaten $x^\mu$ des einen Koordinatensystems
in die Koordinaten $y^\nu$ des anderen Koordinatensystems umrechnet.
Man kann also schreiben
\begin{equation}
y^{\nu}=y^{\nu}(x^1,\dots,x^n).
\label{skript:kruemmung:umrechnung}
\end{equation}
Uns interessiert vor allem die Beschreibung von Bahnkurven, wir möchten
ja zum Beispiel den Absturz in ein schwarzes Loch berechnen können.
Eine Bahnkurve erhält man, indem man die Koordinaten mit der Zeit
varieren lässt.
Eine Kurve wird also beschrieben durch Funktionen $x^\mu(t)$.

In der physikalischen Beschreibung werden meistens Vektoren wie
Geschwindigkeit und Beschleunigung verwendet.
Sie sind die Ableitung der Koordinaten eines Bahnpunktes nach
der Zeit.
Dies lässt sich direkt auch auf die Koordinaten übertragen,
wir erhalten für den Geschwindigkeitsvektor
\[
v^{\mu} = \frac{dx^\mu(t)}{dt}.
\]

Wie sieht der Geschwindigkeitsvektor in $y^{\nu}$ Koordinaten aus?
Dazu setzen wir die Bahnkurve in die Umrechnungsformeln
\eqref{skript:kruemmung:umrechnung} ein.
Die Kettenregel liefert
\begin{align*}
y^{\nu}(t)&=y^{\nu}(x^1(t),\dots,x^n(t))
\\
u^{\nu}
=
\frac{dy^{\nu}(t)}{dt}
&=
\frac{\partial y^{\nu}}{\partial x^1}\frac{dx^1(t)}{dt}
+\dots+
\frac{\partial y^{\nu}}{\partial x^n}\frac{dx^n(t)}{dt}
=
\sum_{\mu=1}^n
\frac{\partial y^{\nu}}{\partial x^{\mu}}\frac{dx^{\mu}(t)}{dt}
\end{align*}
Auf der rechten Seite steht eine Summe von Termen, in denen
der Summationsindex sowohl oben als auch unten auftritt.
Diese Art von Summe kommt in der zu entwickelnden Theorie sehr
häufig vor, daher schreiben wir in Zukunft die Summe nicht mehr.
Diese {\em Einsteinsche Summenkonvention} bedeutet also, dass in
einem Term, in dem der gleiche Index oben und unten vorkommt,
über die möglichen Werte dieses Index summiert werden muss.
\index{Summenkonvention!Einsteinsche}

Die Koeffizienten 
\[
\alpha_\mu^\nu=\frac{\partial y^{\nu}}{\partial x^{\mu}}
\]
dienen also der Umrechnung der Komponenten eines Vektors vom
$x^{\mu}$-Koordinatensystem ins $y^{\nu}$-Koordinatensystem.
Man kann die Rechnung auch in Matrixform schreiben:
\[
\begin{pmatrix}y^1\\\vdots\\y^n\end{pmatrix}
=
\begin{pmatrix}
\frac{\partial y^1}{\partial x^1}&\dots&\frac{\partial y^1}{\partial x^n}\\
\vdots&\ddots&\vdots\\
\frac{\partial y^n}{\partial x^1}&\dots&\frac{\partial y^n}{\partial x^n}
\end{pmatrix}
\begin{pmatrix}x^1\\\vdots\\x^n\end{pmatrix}.
\]
Die Summationskonvention lässt sich zum Beispiel dadurch rechtfertigen,
dass bei der Matrizenmultiplikation die Summation ja auch nicht
explizit hingeschrieben wird.

\subsection{Metrik}
In einem rechtwinkligen Koordinatensystem können wir die Länge
einer Kurve durch Zerlegen in beliebig kleine Teilstücke 
\begin{align*}
l
&\simeq
\sum_{i=1}^n \sqrt{\sum_{\mu} (x^{\mu}(t_i)-x^{\mu}(t_{i-1}))^2}
\\
&\rightarrow
\int_{t_0}^{t_n} \sqrt{\sum_{\mu}\biggl(\frac{dx^{\mu}(t)}{dt}\biggr)^2}\,dt
\end{align*}
bestimmen.
Unter der Wurzel sehen wir die Quadratsummen wieder, die für den
Satz des Pythagoras charakteristisch sind.

In einem beliebigen Koordinatensystem funktioniert dies jedoch nicht
mehr.
Wir müssen zulassen, dass die Koordinaten entlang verschiedener
Achsen nicht mehr direkt der Länge entsprechen, dass wir als
entlang der Achsen Skalierungsfaktoren haben.
Weiter ist damit zu rechnen, dass auch gemischte Termen auftauchen.
Die allgemeinstmögliche Form einer Längenmessungsformel ist daher
\begin{equation}
l
=
\int_{t_0}^{t_1}
\sqrt{\sum_{\mu,\nu} g_{\mu\nu} \frac{dx^{\mu}(t)}{dt}\frac{dx^{\nu}(t)}{dt}}\,dt.
\label{skript:kruemmung:metrikformel}
\end{equation}
Die Zahlen $g_{\mu\nu}$ können dabei auch noch von den Koordinaten
abhängen.
Wir verlangen ausserdem, dass $g_{\mu\nu}=g_{\nu\mu}$ ist, denn
diese beiden Terme sind in \eqref{skript:kruemmung:metrikformel}
auf symmetrische Art und Weise vertreten.

\begin{definition}
Die Zahlen $g_{\mu\nu}$ heissen der metrische Tensor.
\index{Tensor!metrischer}
\end{definition}

%
% Beispiel für Längenmessung in einem Koordinatensystem mit nicht
% orthogonalen Achsen.
%
\begin{beispiel}
Wir untersuchen die Längenmessung mit nicht orthogonalen Achsen.
Statt des gewöhnlichen Koordinatensystems in einer Ebene verwenden
wir die Koordinaten $x'=x$ und $y'=x+y$.
In den ungestrichenen Koordinaten ist der Abstand zwischen zwei
Punkten durch den Satz von Pytagoras
\begin{equation}
l^2 = \Delta x^2 + \Delta y^2
\label{skript:kruemmung:p2}
\end{equation}
gegeben.
Um den Abstand in $x'$-$y'$-Koordinaten auszudrücken, müssen wir diese
erst wieder in $x$-$y$-Koordinaten umrechnen. 
Man findet
\[
x=x'
\qquad\text{und}\qquad
y=y'-x=y'-x'.
\]
Eingesetzt in die Formel \eqref{skript:kruemmung:p2} finden wir
\begin{align*}
l^2
&=
\Delta x^2 + \Delta y^2
=
\Delta x^{\prime 2}
+
(\Delta y'- \Delta x')^2
=
\Delta x^{\prime 2}
+
\Delta y^{\prime 2}-2\Delta x'\Delta y' + \Delta x^{\prime 2}
\\
&= 2 \Delta x^{\prime 2} - 2 \Delta x'\Delta y'+\Delta y^{\prime 2}.
\end{align*}
Die zugehörigen Koeffizienten $g_{\mu\nu}$ sind
\[
g_{11} = 2,\quad
g_{12}=g_{21}=-1\quad\text{und}\quad
g_{22}=1.
\]
Diese Koeffizienten beschreiben die gleiche Längenmessung in den 
gestrichenen Koordinaten wie der Satz von Pythagoras in den ursprünglichen
Koordinaten.
\end{beispiel}

Die Formel \eqref{skript:kruemmung:metrikformel} ist etwas unhandlich.
Wir können aber wieder die Einsteinsche Summenkonvention verwenden,
um das Summenzeichen los zu werden.
Ausserdem kann man die Ableitungen nach $t$ auch mit einem Punkt abkürzen.
Damit lässt sich die Längenmessung daher etwas kompakter als
\[
l=\int_{t_0}^{t_1} \sqrt{g_{\mu\nu}\dot x^{\mu}(t) \dot x^{\nu}(t)}\,dt
\]
schreiben.

\subsection{Beispiele}
Wir betrachten drei für die Anwendungen wichtige Bespiele von
Koordinatensystemen des gewöhnlichen dreidimensionalen Raumes
und berechnen die zugehörigen metrischen Tensoren.

\subsubsection{Polarkoordinaten}
Punkte in der Ebene können statt in rechtwinkligen $x$-$y$-Koordinaten
auch mit Hilfe von Polarkoordinaten $(r,\varphi)$0nach der Umrechnungsregel
\begin{align*}
x&=r\cos\varphi\\
y&=r\sin\varphi
\end{align*}
beschrieben werden.
Um den metrischen Tensor zu bestimmen, müssen die Ableitungen von $x$ 
und $y$ nach $t$ durch Ableitungen von $r$ und $\varphi$ nach $t$ 
ausdrücken.
Die Produktregel liefert:
\begin{align*}
\dot x&= \dot r\cos \varphi - r\dot\varphi \sin\varphi 
\\
\dot y&= \dot r\sin\varphi + r\dot\varphi\cos\varphi.
\end{align*}
Eingesetzt in den Satz von Pythagoras folgt
\begin{align*}
\dot x^2 + \dot y^2
&=
\dot r^2\cos^2\varphi -2r\dot r\dot\varphi\cos\varphi\sin\varphi +r^2\dot \varphi^2\sin^2\varphi
+
\dot r^2\sin^2\varphi +2r\dot r\dot\varphi\sin\varphi\cos\varphi +r^2\dot\varphi^2\cos^2\varphi
\\
&=
\dot r^2(\cos^2\varphi+\sin^2\varphi)+ r^2\dot\varphi^2(\sin^2\varphi+\cos^2\varphi)
\\
&=\dot r^2 + r^2\dot\varphi^2.
\end{align*}
Die gemischten Terme haben sich weggehoben.
Man liest daraus für die Koeffizienten des metrischen Tensors
\[
g_{11}=r^2,\qquad g_{12}=g_{21}=0\qquad\text{und}\qquad g_{22}=r^2\dot\varphi^2
\]
ab.
Die Koeffizienten hängen zwar von den Koordinaten ab, doch bedeutet
das noch nicht, dass ein gekrümmter Raum vorliegt.
Diese $g_{\mu\nu}$ beschreiben ja die gleiche Lägenmessung wie der Satz
des Pythagoras in $x$-$y$-Koordinaten.

\subsubsection{Zylinderkoordinaten}
Zylinderkoordinaten beschreiben die Punkte des dreidimensionalen
Raumes mit Polarkoordinaten in der $x$-$y$-Ebene und der $z$-Koordinate.
Da wir die Metrik in der $x$-$y$-Ebene schon durch $(r,\varphi)$
ausgedrückt haben, können wir auch die Metrik in Polarkoordinaten
bekommen, indem wir die $z$-Koordinaten ergänzen:
\[
\dot x^2+\dot y^2 +\dot z^2
=
\dot r^2 + r^2\dot\varphi^2 + \dot z^2.
\]
Der zugehörige metrische Tensor hat daher die Koeffizienten
\[
g_{11}=\dot r^2,\qquad
g_{22}=r^2\dot\varphi^2
\qquad\text{und}\qquad
g_{33}=1,
\]
alle anderen Koeffizienten sind $0$.

\subsubsection{Zylinderoberfläche}
Beschränken wir uns auf die Punkte im Abstand $1$ zur $z$-Achse, erhalten
wir eine Zylinderfläche, welche mit Koordinaten $\varphi$ und $z$
beschrieben werden kann.
Die Längenmessung in dieser Fläche wird durch den metrischen
Tensor der Zylinderkoordinaten beschrieben, in dem wir $r=1$ einsetzen.
Wir erhalten
\[
\dot\varphi^2+\dot z^2.
\]
Dies ist der metrische Tensor einer Ebene mit rechtwinkligen Koordinaten.

\subsubsection{Kugelkoordinaten}
Kugelkoordinaten beschreiben die Punkte eines dreidimensionalen Raumes
durch die Entfernung $r$ vom Nullpunkt, die geographische Länge
$\varphi$, die von
der $x$-Koordinate gemessen wird, und durch die geograpische Breite
$\vartheta$,
die als Winkel von der $z$-Achse gemessen wird.
Die Umrechnung in kartesische Koordinaten erfolgt mit den Formeln
\begin{align*}
x&= r\sin\vartheta\cos\varphi\\
y&= r\sin\vartheta\sin\varphi\\
z&= r\cos\vartheta
\end{align*}
Wir bestimmen wieder die Koeffizienten des metrischen Tensors.
Dazu leiten wir zunächst nach $t$ ab.
\begin{align*}
\dot x
&=
\dot r\sin\vartheta\cos\varphi
+
r\dot\vartheta \cos\vartheta\cos\varphi
-
r\dot\varphi \sin\vartheta\sin\varphi
\\
\dot y
&=
\dot r\sin\vartheta\sin\varphi
+
r\dot\vartheta\cos\vartheta\sin\varphi
+
r\dot\varphi\sin\vartheta\cos\varphi
\\
\dot z
&=
\dot r\cos\vartheta
-
r\dot\vartheta \sin\vartheta
\end{align*}
Bei der Berechnung der Länge werden sich wieder viele Terme
wegen der verschiedenen Vorzeichen des letzten Terms im
Ausdruck für $\dot x$ und $\dot y$ wegheben.
Für den Ausdruck $ \dot x^2 + \dot y^2 + \dot z^2$ findet man
\[
\begin{array}{clclclcl}
 &
\dot r^2\sin^2\vartheta\cos^2\varphi
	&+&r^2\dot\vartheta^2\cos^2\vartheta\cos^2\varphi
		&+&r^2\dot\varphi^2\sin^2\vartheta\sin^2\varphi
			&+&2r\dot r\dot\vartheta\sin\vartheta\cos\vartheta\cos^2\varphi
\\
+&
\dot r^2\sin^2\vartheta\sin^2\varphi
	&+&r^2\dot\vartheta^2\cos^2\vartheta\sin^2\varphi
		&+&r^2\dot\varphi^2\sin^2\vartheta\cos^2\varphi
			&+&2r\dot r\dot\vartheta\sin\vartheta\cos\vartheta\sin^2\varphi
\\
+&
\dot r^2\cos^2\vartheta
	&+&r^2\dot\vartheta^2\sin^2\vartheta
		& &
			&-&2r\dot r\dot\vartheta \sin\vartheta \cos\vartheta
\\
\end{array}
\]
In den Spalten ergänzen sich in den ersten beiden Zeilen jeweils
$\cos^2\varphi$ und $\sin^2\varphi$ zu $1$.
In den ersten beiden Spalten lassen sich danach auch noch
$\cos^2\vartheta$ und $\sin^2\vartheta$ zu $1$ zusammenfassen,
während sich die Terme in der vierten Spalte wegheben.
Damit bekommt man
\begin{equation}
\dot x^2 + \dot y^2 + \dot z^2
=
\dot r^2+r^2\dot\vartheta^2 + r^2\dot\varphi^2\sin^2\vartheta
\label{skript:kruemmung:kugelkoordinaten}
\end{equation}
für die Längenmessung, und
\[
g_{11}=\dot r^2,\qquad
g_{22}=r^2\dot\vartheta^2
\qquad\text{und}\qquad
g_{33}= r^2\dot\varphi^2\sin^2\vartheta
\]
für die nicht verschwindenden Komponenten des metrischen Tensors.

\subsubsection{Kugeloberfläche}
Aus dem metrischen Tensor der Kugelkoordinaten lässt sich durch festhalten
des Radius der metrische Tensor einer Kugeloberfläche ableiten.
Aus dem Ausruck \eqref{skript:kruemmung:kugelkoordinaten}
erhalten wir
\[
\dot\vartheta^2+\dot\varphi^2\sin^2\vartheta.
\]
Der $\sin$-Term deutet an, dass wir hier nicht mehr direkt eine flache
Metrik haben.
Den Nachweis können wir aber erst führen, wenn wir den Begriff der
Krümmung zur Verfügung haben.
Für die nicht verschwindenden Komponenten des metrischen Tensors
finden wir
\[
g_{11} = 1
\qquad\text{und}\qquad
g_{22}=\sin^2\vartheta.
\]
Man erkennt, dass der Koeffizient $g_{22}$ bei $\vartheta \in\{0,\pi\}$
verschwindet.
Man spricht von einer Singularität der Längenmessung.
Dies ist jedoch nur ein Artefakt der Tatsache, dass die Kugelkoordinaten
bei den Polen nicht mehr eindeutig sind.
Zur Beschreibung der Pole sind alle möglichen Werte der geographischen
Länge gleichermassen geeignet.



%
% k-geodaeten.tex -- Gleichung der Geodäten, Christoffel-Symbole
%
% (c) 2017 Prof Dr Andreas Müller, Hochschule Rapperswil
%
\section{Geodäten
\label{skript:kruemmung:section:geodaeten}}
\rhead{Geodäten}
In der Ebene ist die kürzeste Verbindung zwischen zwei Punkten
eine Gerade.
Auf einem Zylinder oder Kegel kann man die kürzeste Verbindung finden,
indem man die Fläche in eine Ebene abrollt, und dann dort verwendet,
dass die kürzeste Verbindung in der Ebene eine Gerade ist.
Dies zeigt dass die kürzsten Verbindung nichts mit der speziellen
Einbettung einer Fläche zu tun hat, sondern eine Eigenschaft ist,
die sich allein aus der Längenmessung in der Fläche ist, man nennt
dies auch eine intrinsische Eigenschaft.

Auf einer Kugeloberfläche kann man die kürzesten Verbindungen ebenfalls
direkt angeben, es sind die Grosskreise.
Man kann dabei so argumentieren: von allen Schnitten der Kugeloberfläche
mit Ebenen durch die zwei gegeben Punkte ist der Grosskreis derjenige
mit der kleinsten Krümmung, und daher die ``direkteste'' Verbindung.
Dieses Argument ist allerdings nicht ganz exakt, denn man vergisst dabei,
dass es noch viele weitere Kurven gibt, die die beiden Punkte verbinden.
Es ist auch nicht wirklich auf noch allgemeinere Situationen übertragbar,
denn es nützt aus, dass die Kugeloberfläche homogen ist, in jedem Punkt
ist die ``Krümmung'' (ein im Moment noch nicht definierter Begriff) 
gleich gross.

In diesem Abschnitt suchen wir daher nach einer allgemeinen Methode,
die kürzeste Verbindung, die sogenannten Geodäten zu finden.

\subsection{Paralleltransport}
Die Beispiele von kürzesten Verbindungen suggerieren, dass die kürzeste
Verbindung auch die ``geradeste'' ist, sie weicht möglichst wenig von
der Richtung des aktuellen Tangentialvektors ab.
Das Problem bei dieser Interpretation ist allerdings, dass wir Vektoren
in zwei verschiedenen Punkten der Fläche nicht unmittelbar vergleichen
können.
Auf der Kugeloberfläche liegen die Tangentialvektoren an eine Kurve in
verschiedenen Punkten zum Beispiel in verschiedenen Tangentialebenen
an die Kugel.

Wir müssen also zunächst in der Lage sein, Vektoren in zwei verschiedenen
Punkten miteinander zu vergleichen.
Wir können das tun, indem wir einen Vektor entlang einer Kurve transportieren,
wobei wir versuchen, in so parallel zu sich selbst wie möglich zu sich
selbst zu halten.
Auch dies ist im Moment noch ein etwas schwammiger Begriff. 
Es ist aber klar, dass die Komponenten des transportierten Vektors
sowohl von der Transportrichtung wie auch vom ursprünglichen Vektor
abhöngen.
Seien $A^\mu$ die Komponenten eines Vektors, die Richtung mit Komponenten
$\Delta x^\nu$ transport werden soll.
Dann wird der transportierte die Form
\[
\tilde A^\alpha
=
A^\alpha - \Gamma_{\mu\nu}^\alpha A^\mu \Delta x^\nu
\]
haben.
Die Wahl des Vorzeichnes von $\Gamma_{\mu\nu}^\alpha$ ist im Wesentlichen
eine Frage der Konvention.
Die instantane Änderung zur Zeit $t=0$ entlang der Kurve ist
\begin{equation}
\frac{d}{dt}\tilde A^\alpha\bigg|_{t=0}
=
\Gamma_{\mu\nu}^\alpha A^\mu \dot x^\nu.
\label{skript:kruemmung:ableitung}
\end{equation}

Bis jetzt haben wir die Metrik nicht verwendet.
Wir möchten dass der Paralleltransport die Länge des Vektors beim
Transport entlang einer Kurve nicht verändert.
Die Länge des Vektors wird durch $g_{\mu\nu}\tilde A^\mu \tilde A^\nu$
gegeben.
Die Ableitung entlang der Kurve $x^\mu(t)$ ist
\[
\frac{d}{dt} g_{\mu\nu}\tilde A^\mu \tilde A^\nu\bigg|_{t=0}
=
\frac{\partial g_{\mu\nu}}{\partial x^\alpha}A^\mu A^\nu\dot x^\alpha
-
g_{\mu\nu}A^\mu\Gamma_{\alpha\beta}^\nu A^\alpha \dot x^\beta
-
g_{\mu\nu}\Gamma_{\alpha\beta}^\mu A^\alpha \dot x^\beta A^\nu
=
0.
\]
In diesem Ausdruck kommen in allen Termen $A^\mu$ und $\dot x^\beta$ mit
verschiedenen Indizes vor.
Damit wir diese Faktoren ausklammern können, bennen wir die Indizes um,
so dass sie in allen Termen gleich sind.
Wir erhalten dann die Gleichung
\[
\biggl(
\frac{\partial g_{\mu\nu}}{\partial x^\alpha}
-
g_{\mu\beta}\Gamma_{\nu\alpha}^\beta
-
g_{\beta\nu}\Gamma_{\mu\alpha}^\beta
\biggr)
A^\mu A^\nu\dot x^\alpha
=
0
\]
F"ur die Koeffizienten $\Gamma$.
Diese Gleichung muss f"ur jede beliebige Wahl von $A^\mu$ und jede
beliebige Richtung der Kurve $\dot x^\alpha$ erfüllt sein, wenn der
Klammerausdruck verschwindget für alle Werte der freien, also nicht durch
die Summationskonvention als Laufindizes ausgezeichnten, Indizes.
Wir erhalten also
\[
\frac{\partial g_{\mu\nu}}{\partial x^\alpha}
-
g_{\mu\beta}\Gamma_{\nu\alpha}^\beta
-
g_{\beta\nu}\Gamma_{\mu\alpha}^\beta
=
0,
\]
ein System von Gleichungen für die $n^3$ Grössen $\Gamma_{\mu\nu}^\alpha$.
Wir können die $\Gamma$ auch allein auf der linken Seite haben:
\[
g_{\mu\beta}\Gamma_{\nu\alpha}^\beta
+
g_{\beta\nu}\Gamma_{\mu\alpha}^\beta
=
\frac{\partial g_{\mu\nu}}{\partial x^\alpha}
\]
Durch zyklische Vertauschung der drei Indizes $\mu$, $\nu$ und $\alpha$
erhalten wir drei Gleichungen
\[
\def\arraystretch{2.0}
\begin{linsys}{3}
g_{\mu\beta}\Gamma_{\nu\alpha}^\beta &+& g_{\beta\nu}\Gamma_{\mu\alpha}^\beta
& &
&=&
\displaystyle
\frac{\partial g_{\mu\nu}}{\partial x^\alpha}
\\
& &g_{\nu\beta}\Gamma_{\alpha\mu}^\beta &+& g_{\beta\alpha}\Gamma_{\nu\mu}^\beta
&=&
\displaystyle
\frac{\partial g_{\nu\alpha}}{\partial x^\mu}
\\
g_{\beta\mu}\Gamma_{\alpha\nu}^\beta
& &
&+&g_{\alpha\beta}\Gamma_{\mu\nu}^\beta 
&=&
\displaystyle
\frac{\partial g_{\alpha\mu}}{\partial x^\nu}
\end{linsys}
\]
Da sowohl $g$ als auch $\Gamma$ in den unteren Indizes symmetrisch
sind, können wir die Gleichungen weiter vereinfachen:
\[
\def\arraystretch{2.0}
\begin{linsys}{3}
g_{\mu\beta}\Gamma_{\nu\alpha}^\beta
	&+& g_{\nu\beta}\Gamma_{\mu\alpha}^\beta
		& &
&=&
\displaystyle
\frac{\partial g_{\mu\nu}}{\partial x^\alpha}
\\
	& &g_{\nu\beta}\Gamma_{\mu\alpha}^\beta
		&+& g_{\alpha\beta}\Gamma_{\nu\mu}^\beta
&=&
\displaystyle
\frac{\partial g_{\nu\alpha}}{\partial x^\mu}
\\
g_{\mu\beta}\Gamma_{\nu\alpha}^\beta
	& &
		&+&g_{\alpha\beta}\Gamma_{\nu\mu}^\beta 
&=&
\displaystyle
\frac{\partial g_{\alpha\mu}}{\partial x^\nu}
\end{linsys}
\]
Subtrahieren wir die erste Zeile von der Summe der letzten beiden,
heben sich ersten Terme weg, es bleibt
\[
g_{\alpha\beta}\Gamma_{\nu\mu}^\beta
=
\biggl(
\frac{\partial g_{\nu\alpha}}{\partial x^\mu}
+
\frac{\partial g_{\alpha\mu}}{\partial x^\nu}
-
\frac{\partial g_{\mu\nu}}{\partial x^\alpha}
\biggr).
\]
Bezeichen wir die inverse Matrix von $g_{\alpha\beta}$ mit
$g^{\alpha\beta}$, dann k"onnen wir nach $\Gamma_{\mu\nu}^\beta$ aufl"osen:
\[
g^{\sigma\alpha}
\biggl(
\frac{\partial g_{\nu\alpha}}{\partial x^\mu}
+
\frac{\partial g_{\alpha\mu}}{\partial x^\nu}
-
\frac{\partial g_{\mu\nu}}{\partial x^\alpha}
\biggr)
=
g^{\sigma\alpha}
g_{\alpha\beta}\Gamma_{\mu\nu}^\beta
=
\delta^\sigma_\beta\Gamma_{\mu\nu}^\beta
=
\Gamma_{\mu\nu}^\sigma.
\]

\begin{definition}
Sei $g_{\mu\nu}$ ein metrischer Tensor. 
Dann heissen die
\[
\Gamma_{\alpha,\mu\nu}
=
\frac12
\biggl(
\frac{\partial g_{\nu\alpha}}{\partial x^\mu}
+
\frac{\partial g_{\alpha\mu}}{\partial x^\nu}
-
\frac{\partial g_{\mu\nu}}{\partial x^\alpha}
\biggr)
\]
die {\em Christoffelsymbole 1.~Art}
und
\[
\Gamma_{\mu\nu}^\sigma
=
g^{\sigma\alpha} \Gamma_{\alpha,\mu\nu}
=
\frac12
g^{\sigma\alpha}
\biggl(
\frac{\partial g_{\nu\alpha}}{\partial x^\mu}
+
\frac{\partial g_{\alpha\mu}}{\partial x^\nu}
-
\frac{\partial g_{\mu\nu}}{\partial x^\alpha}
\biggr)
\]
heissen {\em Christoffelsymbole 2.~Art} oder {\em Zusammenhangskoeffizienten}.
\end{definition}

\subsection{Beispiele}
In den nachfolgenden Beispielen wollen wir die Christoffelsymbole erster
und zweiter Art für Zylinder- und Kugeloberfläche berechnen.

\subsubsection{Polarkoordinaten}
Die nicht verschwinden Komponenten
des metrischen Tensors in Polarkoordinaten $(r,\varphi)$
sind $g_{11}=1$ und $g_{22}=r^2$.
Davon brauchen wir die Ableitungen
\[
\begin{aligned}
\frac{\partial g_{11}}{\partial x^1} &=0,&
\frac{\partial g_{12}}{\partial x^1} &=0,&
\frac{\partial g_{21}}{\partial x^1} &=0,&
\frac{\partial g_{22}}{\partial x^1} &=2r,&
\\
\frac{\partial g_{11}}{\partial x^2} &=0,&
\frac{\partial g_{12}}{\partial x^2} &=0,&
\frac{\partial g_{21}}{\partial x^2} &=0,&
\frac{\partial g_{22}}{\partial x^2} &=0
\end{aligned}
\]
Die Christoffelsymbole erster Art sind daher 
\[
\begin{aligned}
\Gamma_{1,11} &=  0,&
\Gamma_{1,12} &=  0,&
\Gamma_{1,21} &=  0,&
\Gamma_{1,22} &= -1,
\\
\Gamma_{2,11} &=  0,&
\Gamma_{2,12} &=  1,&
\Gamma_{2,21} &=  1,&
\Gamma_{2,22} &=  0
\end{aligned}
\]
und die Christoffelsymbole zweiter Art sind
\[
\begin{aligned}
\Gamma_{11}^1 &= 0,&
\Gamma_{12}^1 &= 0,&
\Gamma_{21}^1 &= 0,&
\Gamma_{22}^1 &=-1,
\\
\Gamma_{11}^2 &= 0,&
\Gamma_{12}^2 &= \frac1{r^2},&
\Gamma_{21}^2 &= \frac1{r^2},&
\Gamma_{22}^2 &= 0.
\end{aligned}
\]

\subsubsection{Zylinderkoordinaten}
Da die Komponenten des metrischen Tensors sind konstant, damit verschwinden
alle Ableitungen
\[
\frac{\partial g_{\mu\nu}}{\partial x^\alpha}=0
\qquad\Rightarrow\qquad
\Gamma_{\alpha,\mu\nu}=0
\qquad\Rightarrow\qquad
\Gamma_{\mu\nu}^\alpha=0.
\]

\subsubsection{Kugelkoordinaten}
Die Kugeloberfläche verwendet die Koordinaten $(\vartheta,\varphi)$.
Zun"achst brauchen wir die Ableitungen der Komponenten des metrischen
Tensors
\begin{align*}
\frac{\partial g_{11}}{\partial \vartheta} &=0,
&
\frac{\partial g_{12}}{\partial \vartheta} &=0,
&
\frac{\partial g_{21}}{\partial \vartheta} &=0,
&
\frac{\partial g_{22}}{\partial \vartheta} &=\sin2\vartheta,
\\
\frac{\partial g_{11}}{\partial \varphi} &=0,
&
\frac{\partial g_{12}}{\partial \varphi} &=0,
&
\frac{\partial g_{21}}{\partial \varphi} &=0,
&
\frac{\partial g_{22}}{\partial \varphi} &=0.
\\
\end{align*}
Daraus können wir die Christoffelsymbole erster Art ableiten:
\begin{align*}
 \Gamma_{1,11}
&=
\frac12\biggl(\frac{\partial g_{11}}{\partial \vartheta}
	+ \frac{\partial g_{11}}{\partial \vartheta}
	- \frac{\partial g_{11}}{\partial \vartheta}\biggr)=0,
&\Gamma_{1,12}
&=
\frac12\biggl(\frac{\partial g_{11}}{\partial \varphi}
	+ \frac{\partial g_{21}}{\partial \vartheta}
	- \frac{\partial g_{12}}{\partial \vartheta}\biggr)=0,
\\
\Gamma_{1,21}
&=
\frac12\biggl(\frac{\partial g_{12}}{\partial \vartheta}
	+ \frac{\partial g_{11}}{\partial \varphi}
	- \frac{\partial g_{21}}{\partial \vartheta}\biggr)=0,
&\Gamma_{1,22}
&=
\frac12\biggl(\frac{\partial g_{12}}{\partial \varphi}
	+ \frac{\partial g_{12}}{\partial \varphi}
	- \frac{\partial g_{22}}{\partial \vartheta}\biggr)=-\frac12\sin2\vartheta,
\\
\Gamma_{2,11}
&=
\frac12\biggl(\frac{\partial g_{12}}{\partial \vartheta}
	+ \frac{\partial g_{12}}{\partial \vartheta}
	- \frac{\partial g_{11}}{\partial \varphi}\biggr)=0,
&\Gamma_{2,12}
&=
\frac12\biggl(\frac{\partial g_{12}}{\partial \varphi}
	+ \frac{\partial g_{22}}{\partial \vartheta}
	- \frac{\partial g_{12}}{\partial \varphi}\biggr)=\frac12\sin2\vartheta,
\\
\Gamma_{2,21}
&=
\frac12\biggl(\frac{\partial g_{12}}{\partial \varphi}
	+ \frac{\partial g_{22}}{\partial \vartheta}
	- \frac{\partial g_{21}}{\partial \varphi}\biggr)=\frac12\sin2\vartheta,
&\Gamma_{2,22}
&=
\frac12\biggl(\frac{\partial g_{22}}{\partial \varphi}
	+ \frac{\partial g_{22}}{\partial \varphi}
	- \frac{\partial g_{22}}{\partial \varphi}\biggr)=0.
\end{align*}
Die inverse Matrix von $g_{\mu\nu}$ hat die nicht verschwindenen
Komponenten
\[
g^{11} = 1
\qquad\text{und}\qquad
g^{22} = \frac1{\sin^2\vartheta}
\]
und die Christoffelsymbole 2.~Art
\begin{equation}
\begin{aligned}
 \Gamma_{11}^1
&=0,
&\Gamma_{12}^1
&=0,
&\Gamma_{21}^1
&=0,
&\Gamma_{22}^1
&=-\frac12\sin2\vartheta,
\\
 \Gamma_{11}^2
&=0,
&\Gamma_{12}^2
&=\cot\vartheta,
&\Gamma_{21}^2
&=\cot\vartheta,
&\Gamma_{22}^2
&=0.
\end{aligned}
\label{skript:kruemmung:christoffelkugel}
\end{equation}

\subsection{Geodätengleichung}
Die Rolle der Geraden in der Ebene müssen die jenigen Kurven auf der
Fläche übernehmen, die so gerade wie möglich sind. 
Eine Gerade ist dadurch charakterisiert, dass sie überall die gleiche
Richtung hat. 
Diese Formulierung ist aber nur möglich, weil Tangentialvektoren in
beliebigen Punkten unmittelbar miteinander vergleichen können.
Für einen allgemeinen Raum müssen wir diesen Vergleich mit Hilfe
des Paralleltransportes durchführen.
Die Forderung an die Kurve wird dann, dass der Tangentialvektor
an die Kurve durch Paralleltransport wieder in einen Tangentialvektor
an die Kurve übergeht.

Etwas formaler setzen wir den Tangentialvektor $\dot x^\mu$ an die
Kurve in die Gleichung \eqref{skript:kruemmung:ableitung} ein und
erhalten die Differentialgleichung 
\begin{equation}
\ddot x^\alpha=\Gamma_{\mu\nu}^\alpha \dot x^\mu\dot x^\nu.
\label{skript:kruemmung:geodatengleichung}
\end{equation}
\index{Geodäte}

\begin{definition}
Eine Kurve in einem Riemannschen Raum heisst eine Geodäte, wenn sie
die Differentialgleichung~\eqref{skript:kruemmung:geodatengleichung}
erfüllt.
\end{definition}

In den nachfolgenden Beispielen wollen wir die Geodäten für diejenigen
Räume berechnen, für die wir die Christoffelsymbole bereits bestimmt
haben.

%
% XXX Geschwindigkeitsinvarianz
%

\subsubsection{Flache R"aume}
In flachen Räumen wie der Ebene oder der Zylinderoberfläche verschwinden
alle Christoffelsymbole.
Die Geodätengleichung ist daher nur nocht
$\ddot x^\alpha=0$, was gleichbedeutend ist mit
\[
x^\alpha(t)=x^\alpha(0) + t \dot x^\alpha(0),
\]
also einer Geradengleichung.
In flachen Räumen sind die Geodäten Geraden.

\subsubsection{Polarkoordinaten}
Die Geodätengleichungen in Polarkoordinante sind
\begin{align*}
\ddot r &= -\dot \varphi^2,
\\
\ddot \varphi &= 2\biggl(\frac{\dot r}{r}\biggr)^2.
\end{align*}

\subsubsection{Kugeloberfläche}
Die Christoffelsymbole zweiter Art für die Kugeloberfläche haben wir
in \eqref{skript:kruemmung:christoffelkugel} berechnet.
Statt die Differentialgleichung der Geodäten direkt zu lösen, versuchen
wir nur, die Parameterdarstellung eines Grosskreises in die
Differentialgleichung einzusetzen und damit zu zeigen, dass die
Grosskreise Lösungen der Geodätengleichung sind.
Da es durch jeden Punkt der Kugeloberflächen und zu jeder Tangentialrichtung
einen Grosskreis gibt, und die Lösungen der Geodätengleichung eindeutig
sind, können schliessen, dass alle Geodäten Grosskreise sind.

Der Äquator der Kugel hat eine besonders einfache Parametrisierung mit
\[
\begin{aligned}
x^1(t)=\vartheta(t)&=\frac{\pi}2,
&\qquad&&
x^2(t)=\varphi(t)&=t
\end{aligned}
\]
mit den Ableitungen
\[
\begin{aligned}
\dot x^1(t)&=0,
&\qquad&&
\dot x^2(t)&=1.
\end{aligned}
\]
Die zweiten Ableitungen der Koordinaten verschwinden, da die ersten
Ableitungen konstant sind.
Wir setzen die ersten Ableitungen in die Geodätengleichung ein.
Es bleiben nur die Terme mit $\mu=\nu=2$ stehen, da $\dot x^1=0$ ist,
also
\begin{align}
\ddot x^1
&=
\Gamma_{\mu\nu}^1\dot x^\mu\dot x^\nu=\Gamma_{22}^1=-\frac12\sin2\theta
=
-\frac12\sin\biggl( 2\cdot\frac{\pi}2\biggr)
=
-\frac12\sin\pi=0,
\label{skript:kruemmung:geodaete:breitenkreis}
\\
\ddot x^2
&=
\Gamma_{\mu\nu}^2\dot x^\mu\dot x^\nu=\Gamma_{22}^2=0.
\notag
\end{align}
Die Geodätengleichung ist für den Äquator erfüllt, der Äquator ist
eine Geodäte.

An dieser Stelle könnten wir mit den Rechnungen eigentlich aufhören, 
denn jeder andere Grosskreis entsteht aus dem Äquator durch eine
Drehung des dreidimensionalen Raumes.
Die Eigenschaft einer Kurve, eine Geodäte zu sein, hängt aber nur von
der Metrik ab, die sich bei einer solchen Drehung nicht ändert.
Jeder andere Grosskreis ist also automatisch auch eine Geodäte.
Trotzdem wollen wir im folgenden für jeden beliebigen Grosskreis
zeigen, dass er die Differentialgleichung der Geodäten erfüllt.

Ein Breitenkreis ist nach der gleichen Rechnung keine Geodäte.
Er ist charakterisiert durch $x^(t)=\vartheta\ne\frac{\pi}2$,
die Differentialgleichung
\eqref{skript:kruemmung:geodaete:breitenkreis}
wird damit zu
\[
\ddot x^1
=
\Gamma_{\mu\nu}^1\dot x^\mu\dot x^\nu=\Gamma_{22}^1=-\frac12\sin2\theta
\ne
0,
\]
die Differentialgleichung der Geodäten ist nicht erfüllt.
Die rechte Seite ist sogar konstant, dies besagt dass der Breitenkreis
im Bezug zu einer Geodäten so gekrümmt ist, dass er ganz
auf einer Seite des Grosskreises mit gleichem Anfangspunkt und
Anfangsgeschwindigkeit liegt.

Als nächstes betrachten wir die Meridiane der Kugel.
Eine Merian hat die Parameterdarstellung
\[
\begin{aligned}
x^1(t)=\vartheta(t)&=t,
&\qquad&&
x^2(t)=\varphi(t)&=0.
\end{aligned}
\]
Die Tangentialvektoren sind daher
\[
\begin{aligned}
\dot x^1(t)&=1,
&\qquad&&
\dot x^2(t)&=0.
\end{aligned}
\]
Wiederum verschwinden
die zweiten Ableitungen von $x^\mu$.
Wir setzen dies jetzt in die Geodäten\-gleichung ein und erhalten
\begin{align*}
\ddot x^1
&=
\Gamma_{\mu\nu}\dot x^\mu \dot x^\nu
=
\Gamma_{22}^1\dot x^2 \dot x^2=0,
\\
\ddot x^2
&=
\Gamma_{\mu\nu}^2\dot x^\mu \dot x^\nu
=
\Gamma_{11}^2\dot x^1\dot x^1=\Gamma_{11}^2=0.
\end{align*}
Die Differentialgleichungen für eine Geodäte sind für Meridiane erfüllt,
damit ist gezeigt, dass Meridiane Geodäten sind.

Die Berechnung für einen beliebigen Grosskreis ist dagegen sehr 
kompliziert und nicht sehr instruktiv.

\subsection{Variationsprinzip}







%
% k-kruemmung.tex -- Krümmungstensor, Ricci, Einstein
%
% (c) 2017 Prof Dr Andreas Müller, Hochschule Rapperswil
%
\section{Krümmung
\label{skript:kruemmung:section:kruemmung}}
Transportiert man einen Vektor in der Ebene mit dem üblichen euklidischen
Koordinatensysteme parallel, dann ändert seine Richtung nicht,
denn die Christoffelsymbole verschwinden alle.
Auf einer Kugeloberfläche sieht das ganz anders aus.
Transportiert man einen Vektor tangential an den Äquator zunächst entlang 
des Äquators über einen Winkel von $90^\circ$, dann auf einem Längenkreis
bis zum Nordpol und wieder zurück zum Ausgangspunkt.
Wie in Abbildung~\ref{weissnicht} sichtbar, dreht sich der Vektor
dabei um $90^\circ$. 
Der Unterschied rührt natürlich daher, dass die Kugeloberfläche gekrümmt
ist.
Offenbar ist die Änderung der Richtung eines Tangentialvektors beim
Paralleltransport entlang eines geschlossenen Weges ein Mass für die
Krümmung einer Fläche.

In diesem Kapitel wollen wir zeigen, wie aus dem Konzept des
Paralleltransportes ein mathematisch wohldefiniertes Mass für die
Krümmung gewonnen werden kann.

\subsection{Krümmungstensor}
Wir möchten Berechnen, wie sich ein Vektor beim Paralleltransport entlang
einer geschlossenen Kurve ändert.
In dieser Form ist das Problem sicher zu kompliziert, die Wahl
einer geschlossenen Kurve beinhaltet viel zu viele Freiheitsgrade.

Zu zwei Tangentialvektoren $u^\mu$ und $v^\mu$ und in einem Punkt
$P$ können wir immer eine
Fläche finden, die aus Geodäten besteht, die alle vom Punkt $P$ ausgehen
und dort eine Richtung haben, die eine Linearkombination der beiden
Tangentialvektoren ist.
Mit diesem Trick können wir das Problem auf eine zweidimensionale
Fläche reduzieren.
Und statt eine beliebige Kurve zuzulassen, können wir uns weiter
auf einen Polygonzug beschränken, bei dem wir den Geodäten folgen,
die als Tangentialrichtung die Richtung der beiden gegebenen
Tangentialvektoren haben.

Wenn wir einen Vektor $x^\mu$ entlang einer solchen Kurve parallel
transportieren, dann aber die Kurve auf einen Punkt zusammenschrumpfen
lassen, dann entsteht im Grenzwert ein Vektor $y^\mu$,
der die Verschiebung des Vektors $x^\mu$ beschreibt.
Dieser Vektor muss linear von $u^\mu$, $v^\mu$ und $x^\mu$ abhängen,
wir erwarten also, dass in jedem Punkt Zahlen
$R^\alpha\mathstrut_{\mu\nu\sigma}$ geben muss, mit denen man
$y^\mu$ berechnen kann:
\[
y^\alpha = R^\alpha\mathstrut_{\mu\nu\sigma}u^\mu v^\nu x^\sigma.
\]

\subsection{Ricci-Krümmung}





