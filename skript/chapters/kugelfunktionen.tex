%
% m-kugelfunktionen.tex
%
% (c) 2017 Prof Dr Andreas Müller, Hochschule Rapperswil
%
\chapter{Kugelfunktionen%
\label{skript:chapter:kugelfunktionen}}
\lhead{Kugelfunktionen}
\rhead{}
Im vorangegangenen Kapitel~\ref{skript:chapter:multipol} haben wir
die Multipol-Entwicklung kennengelernt. 
Die wesentliche Erkenntnis dabei war, dass wir eine Funktion immer zerlegen
können in eine Summe von Termen der Form
\[
\frac{p(x,y,z)}{r^k},
\]
wobei $p(x,y,z)$ ein homogenes Polynom in den Koordinaten $x$, $y$ und
$z$ war.
Nehmen wir an,d ass das Polynom homogen vom Grad $l$ ist, dann kann man
auch schreiben
\[
\frac{p(x,y,z)}{r^k}
=
p\biggl(\frac{x}{r},\frac{y}{r},\frac{z}{r}\biggr)\frac1{r^{k-l}}.
\]
Die Brüche $x/r$, $y/r$ und $z/r$ sind konstant entlang eines vom
Nullpunkt ausgehenden Strahls, das Polynom 
\[
p\biggl(\frac{x}{r},\frac{y}{r},\frac{z}{r}\biggr)
\]
ist also eine Funktion, die nur von der Richtung des Strahls
abhängt.
Das Potential lässt sich also schreiben als eine Summe von Termen,
die Produkte sind einer Funktion, die nur von der Richtung abhängig
ist, und einer Funktion, die die Entfernungsabhängigkeit vom Nullpunkt
ausdrückt.

In diesem Abschnitt wollen wir zunächst zeigen, dass man die klassische
Fourier-Theorie genau auf die gleiche Art betrachten kann.
Eine Funktion in der Ebene lässt sich immer beschreiben als eine Summe
von Produkten einer Funktion, die die Richtungsabhängigkeit ausdrückt, und
und einer Funktion, die die Entfernungsabhängigkeit codiert.

Die Wahl einer Basis solcher Funktionen ist jedoch nicht eindeutig.
Wenn wir zusätzlich verlangen, dass die Funktionen im Sinne eines noch
zu definierenden Skalarproduktes orthonormiert sind, dann ist die
Berechnung der entsprechenden Koeffizienten besonders einfach und
führt auf die klassische Fourier-Theorie.

Im letzten Abschnitt wollen wir dann zeigen, dass dasselbe Programm
auch im $\mathbb R^3$ durchführbar ist, und dann auf die Kugelfunktionen
führt.

\section{Fourier-Theorie}
Betrachten wir eine Funktion $f(x,y)$ in der Ebenen $(x,y)\in\mathbb R^2$.
Aus Kapitel~\ref{skript:chapter:multipol} wissen wir, dass sich die Funktion
$f(x,y)$ schreiben lässt als Summe von Termen der Form $p(x,y)r^k$, wobei
$p(x,y)$ ein homogenes Polynom ist.
Im Moment interessiert uns nur die Richtungsabhängigkeit, so dass wir
die Funktion $f(x,y)$ auf einen Kreis mit Radius $r$ einschränken können.
Ein homogenes Polynom $p(x,y)$ vom Grad $l$ erfüllt
\[
p(x,y) = p\biggl(\frac{x}{r},\frac{y}{r}\biggr) r^l.
\]
In Polarkoordinaten ist
\[
\begin{aligned}
\frac{x}{r}&=\cos\varphi
&&\text{und}&
\frac{y}{r}&=\sin\varphi,
\end{aligned}
\]
so dass wir die Funktion $p(x,y)$ in Polarkoordinaten schreiben
können als
\[
p(x,y)=r^l p(\cos\varphi,\sin\varphi).
\]
Für eine Funktion $f(x,y)$ auf einem Kreis um dem Nullpunkt erwarten wir
daher, dass wir sie in Polarkoordinaten schreiben können also eine
Linearkombination Funktionen der Form
\begin{equation}
\cos^k\varphi \sin^{l-k}\varphi,\qquad 0\le k\le l.
\label{skript:kugelfunktionen:produkte}
\end{equation}
Die Potenzen machen die Ausdrücke etwas kompliziert, doch gibt
es eine Reihe von Identitäten für trigonometrische Funktionen,
die erlauben, solche Produkte auf Summen von Funktionen von 
Vielfachen des Winkels zu reduzieren.
Dazu gehören einerseits die Formeln für die Potenzen:
\begin{align*}
\cos^n\vartheta
&=
\begin{cases}
\displaystyle
\frac{2}{2^n}\sum_{k=0}^{\frac{n-1}2} \binom{n}{k}\cos((n-2k)\vartheta)
&\qquad\text{$n$ ungerade}
\\[10pt]
\displaystyle
\frac{1}{2^n}\binom{n}{\frac{n}2}
+
\frac{2}{2^n}\sum_{k=0}^{\frac{n}2-1}\cos((n-2k)\vartheta)
&\qquad\text{$n$ gerade}
\end{cases}
\\
\sin^n\vartheta
&=
\begin{cases}
\displaystyle
\frac{2}{2^n}\sum_{k=0}^{\frac{n-1}2} (-1)^{\frac{n-1}2-k}\binom{n}{k}\sin((n-2k)\vartheta)
&\qquad\text{$n$ ungerade}
\\[10pt]
\displaystyle
\frac{1}{2^n}\binom{n}{\frac{n}2}
+
\frac{2}{2^n}\sum_{k=0}^{\frac{n}2-1}(-1)^{\frac{n}2-k}\binom{n}{k}\cos((n-2k)\vartheta)
&\qquad\text{$n$ gerade}
\end{cases}
\end{align*}
Ersetzt man die Potenzen in~\eqref{skript:kugelfunktionen:produkte}
durch diese Ausdrücke, entstehen immer noch Produkte von jeweils
zwei trigonometrischen Funktionen.
Solche Produkte können aber immer ersetzt werden durch eine
Linearkombination von Funktionen dank der Summen- und Differenzenformeln:
\begin{align*}
\cos\vartheta\cos\varphi
&=
\frac12\bigl(\cos(\vartheta-\varphi)+\cos(\vartheta+\varphi)\bigr)
\\
\sin\vartheta\sin\varphi
&=
\frac12\bigl(\cos(\vartheta-\varphi)-\cos(\vartheta+\varphi)\bigr)
\\
\sin\vartheta\cos\varphi
&=
\frac12\bigl(\sin(\vartheta-\varphi)+\sin(\vartheta+\varphi)\bigr)
\\
\cos\vartheta\sin\varphi
&=
\frac12\bigl(\sin(\vartheta-\varphi)-\sin(\vartheta+\varphi)\bigr)
\end{align*}
Damit ist gezeigt, dass die Funktion $p(\cos\varphi,\sin\varphi)$
geschrieben werden kann als Linearkombination von trigonometrischen
Funktionen von Vielfachen des Winkels:
\[
p(x,y)=p(\cos\varphi,\sin\varphi)
=
\frac{a_0}2
+
\sum_{k=1}^l \bigl(
a_k\cos k\varphi+b_k\sin k\varphi
\bigr).
\]
Wir vermuten daher, dass sich auch die Funktion $f(x,y)$ als Summe
\begin{equation}
f(x,y)
=
\frac{a_0}2
+
\sum_{k=1}^\infty \bigl(
a_k\cos k\varphi+b_k\sin k\varphi
\bigr)
\label{skript:kugelfunktionen:fourierreihe}
\end{equation}
schreiben lässt.

\subsection{Orthogonalität}
Die Funktionen $\cos k\varphi$ und $\sin k\varphi$ sind offenbar 
wesentlich besser geeignet, eine Funktion von $\varphi$ auszudrücken,
als die Potenzen $\sin^k\varphi$ und $\cos^k\varphi$ und ihre Produkte.
Dies wird noch unterstütz durch die folgende Beobachtung.
Zunächst schreiben wir
\begin{align*}
c_0(\varphi)&=\frac{1}{\sqrt{2}}
\\
c_k(\varphi)&=\cos k\varphi
\\
s_k(\varphi)&=\sin k\varphi
\end{align*}
für die Funktionen, die in der
Reihe~\eqref{skript:kugelfunktionen:fourierreihe}
benötigt werden.
Ausserdem verwenden wir als Skalarprodukt von Funktionen
den Ausdruck
\begin{equation}
\langle f,g\rangle
=
\frac1{\pi}
\int_{-\pi}^{\pi}
f(\varphi)g(\varphi)
\,d\varphi.
\end{equation}
Dann kann man ausrechnen, dass die Funktionen $c_k$ und $s_k$ im
folgenden Sinne orthonormiert sind.
Verschiedene Funktionen haben Skalarprodukt $0$:
\begin{align*}
\langle c_0,c_k\rangle
&=
\frac{1}{\pi}\int_{-\pi}^\pi \frac{1}{\sqrt{2}}\cos k\varphi\,d\varphi=0
\\
\langle c_0,s_k\rangle
&=
\frac{1}{\pi}\int_{-\pi}^\pi \frac{1}{\sqrt{2}}\sin k\varphi\,d\varphi=0
\\
\langle c_k,c_l\rangle
&=
\frac1{\pi}
\int_{-\pi}^{\pi}
\cos k\varphi\cos l\varphi
\,d\varphi
=
\frac1{\pi}
\int_{-\pi}^{\pi}
\cos(k-l)\varphi + \cos(k+l)\varphi
\,d\varphi
=0
\\
\langle s_k,s_l\rangle
&=
\frac1{\pi}
\int_{-\pi}^{\pi}
\sin k\varphi\sin l\varphi
\,d\varphi
=
\frac1{\pi}
\int_{-\pi}^{\pi}
\cos(k-l)\varphi - \cos(k+l)\varphi
\,d\varphi
=0
\\
\langle s_k,c_l\rangle
&=
\frac1{\pi}
\int_{-\pi}^{\pi}
\cos k\varphi\sin l\varphi
\,d\varphi
=
\frac1{\pi}
\int_{-\pi}^{\pi}
\sin(k-l)\varphi - \sin(k+l)\varphi
\,d\varphi
=0
\end{align*}
Identische Funktionen haben Skalarprodukt $1$ mit sich selbst:
\begin{align*}
\langle c_0,c_0\rangle
&=
\frac1{\pi}\int_{-\pi}^{\pi}
\frac1{\sqrt{2}}
\cdot
\frac1{\sqrt{2}}
\,d\varphi = \frac{2\pi}{\pi}\frac12=1
\\
\langle c_k,c_k\rangle
&=
\frac1{\pi}
\int_{-\pi}^{\pi}
\cos^2 k\varphi
\,d\varphi
=
\frac1{\pi}\cdot\pi = 1
\\
\langle s_k,s_k\rangle
&=
\frac1{\pi}
\int_{-\pi}^\pi
\sin^2 k\varphi
\,d\varphi
=\frac1{\pi}\cdot\pi = 1.
\end{align*}
Die Funktionen $c_k$ und $s_k$ verhalten sich also in jeder Beziehung
wie orthonormierte Basisvektoren in einem endlichdimensionalen Vektorraum.
In abgekürzter Form können wir diese Eigenschaften auch als
\begin{equation}
\begin{aligned}
\langle c_k,c_l\rangle
&=
\delta_{kl}
\\
\langle c_k,s_l\rangle
&=0
\\
\langle s_k,s_l\rangle
&=
\delta_{kl}
\end{aligned}
\label{skript:kugelfunktionen:ortho}
\end{equation}
schreiben, wobei wie früher
\[
\delta_{kl}=\begin{cases}
1&\qquad k=l\\
0&\qquad\text{sonst}
\end{cases}
\]
ist.
Die Beziehungen~\eqref{skript:kugelfunktionen:ortho} heissen die
Orthogonalitätsrelationen für die Funktioenn $c_k$ und $s_k$.
\index{Orthogonalitätsrelationen}

\subsection{Bestimmung der Koeffizienten $a_k$ und $b_k$}
Die Orthogonalitätsrelationen~\eqref{skript:kugelfunktionen:ortho} erlauben
uns, die Koeffizienten $a_k$ und $b_k$ in der
Reihenentwicklung~\eqref{skript:kugelfunktionen:fourierreihe}
zu finden.
Dazu nehmen wir an, dass sich die Funktion $f(\varphi)$ als Reihe
in den Funktionen $c_k$ und $s_k$ ausdrücken lässt.
Wir setzen also $f(\varphi)$ in der Form
\begin{align*}
f(\varphi)
&=
a_0 c_0(\varphi)
+
\sum_{k=1}^\infty \bigl(a_kc_k(\varphi) + b_ks_k(\varphi)\bigr)
\end{align*}
an und bestimmen die Skalarprodukte
$\langle c_k,f\rangle$ und
$\langle s_k,f\rangle$ wie folgt
\begin{align*}
\langle c_k,f\rangle
&=
\biggl\langle
c_k,
a_0 c_0+\sum_{l=1}^\infty\bigl(a_l c_l + b_l s_l\bigr)
\biggr\rangle
=
a_0\langle c_k,c_0\rangle
+
\sum_{l=1}^\infty a_l\langle c_k,c_l\rangle
=
a_k
\\
\langle s_k,f\rangle
&=
\biggl\langle
s_k,
a_0c_0+\sum_{l=1}^\infty (a_lc_l+b_ls_l)
\biggr\rangle
=
\sum_{l=1}^\infty b_l\langle s_k,s_l\rangle
=
b_k
\end{align*}
Unter Verwendung der Definition des Skalarprodukts $\langle\;,\;\rangle$
finden wir jetzt die Formeln:
\begin{align*}
a_0
&=
\frac1{\pi}\int_{-\pi}^{\pi} f(\varphi)\frac1{\sqrt{2}}\,d\varphi
\\
a_k
&=
\frac1{\pi}\int_{-\pi}^{\pi} f(\varphi)\cos k\varphi\,d\varphi
\\
b_k
&=
\frac1{\pi}\int_{-\pi}^{\pi} f(\varphi)\sin k\varphi\,d\varphi
\end{align*}
Der erste Term der Reihe wird dann
\begin{align*}
a_0c_0(\varphi)
&=
\frac1{\pi}\int_{-\pi}^{\pi} f(\varphi)\frac1{\sqrt{2}}\,d\varphi
\cdot
\frac1{\sqrt{2}}
=
\frac12
\cdot
\frac1{\pi}\int_{-\pi}^{\pi} f(\varphi) \,d\varphi.
\end{align*}
Es ist daher üblich, den Koeffizienten $a_0$ mit der gleichen
Formel wie die $a_k$ für $k>0$ zu berechnen.
Wir vermuten daher den folgenden Satz
\begin{satz}
Eine $2\pi$-periodische stetige Funktion $f(\varphi)$ kann geschrieben
werden als
\begin{equation}
\frac{a_0}2
+
\sum_{k=1}^\infty\bigl(
a_k\cos k\varphi + b_k\sin k\varphi
\bigr),
\label{skript:kugelfuntionen:satz:reihe}
\end{equation}
wobei die Koeffizienten $a_k$ und $b_k$ mit den Formeln
\begin{equation}
\begin{aligned}
a_k
&=
\frac1{\pi}
\int_{-\pi}^\pi f(\varphi)\cos k\varphi\,d\varphi
&
&\text{für $k\ge 0$}
\\
b_k
&=
\frac1{\pi}
\int_{-\pi}^\pi f(\varphi)\sin k\varphi\,d\varphi
&
&\text{für $k>0$}
\end{aligned}
\label{skript:kugelfuntionen:satz:koeffizienten}
\end{equation}
berechnet werden können.
\end{satz}

Bisher haben wir nur gezeigt, dass eine solche Formel plausibel ist.
Wenn die Funktion $f(\varphi)$ als endliche Reihe der
Form~\eqref{skript:kugelfunktionen:fourierreihe}, dann trifft der
Satz zu, wie obige Rechnung gezeigt hat.
Für beliebige stetige oder noch allgemeinere Funktionen muss
die Konvergenz der Reihe~\eqref{skript:kugelfuntionen:satz:reihe}
gebildet mit den
Koeffizienten~\eqref{skript:kugelfuntionen:satz:koeffizienten}
bewiesen werden muss.

\section{Kugelfunktionen}
Die Betrachtungen zur Fouriertheorie haben uns gezeigt, dass Funktionen
auf einem Kreis, also $2\pi$-periodische Funktionen, durch eine
Fourierreihe dargestellt werden können.
Die Basisfunktionen der Forierreihen haben sich aus den
Funktionen $x/r$ und $y/r$ ergeben, indem wir $x/r=\cos\varphi$
und $y/r=\sin\varphi$ gesehen haben.

Jetzt wollen wir diese Ideen auf die dreidimensionale Situation
übertragen.
Wir interessieren uns also für Funktionen auf einer Kugeloberfläche,
die wir der Einfachheit halber als Einheitskugel annehmen können.
Als Koordinatensystem auf der Kugel verwenden wir Kugelkoordinaten mit
\begin{equation}
\frac1r
\begin{pmatrix}
x\\y\\z
\end{pmatrix}
=
\begin{pmatrix}
\sin\vartheta\cos\varphi\\
\sin\vartheta\sin\varphi\\
\cos\vartheta
\end{pmatrix}.
\end{equation}
Die Abhängigkeit von $\varphi$ kann natürlich wie bei einer Funktion
auf einem Kreis durch die Fourier-Theorie dargestellt werden.
Die Abhängigkeit von $z/r=\cos\vartheta$ muss sich als ein Polynom
in $\cos\vartheta$ ausdrücken lassen.

\subsection{Legendre-Polynome}
Es müssen also Polynome $P_m^l(z)$ gefunden werden derart, dass 
die Funktionen
% XXX Formel für Kugelfunktionen
orthonormiert sind.













