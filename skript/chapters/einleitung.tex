%
% einleitung.tex -- Einleitung zum Skript ueber Differentialgleichungen
%
% (c) 2015 Prof Dr Andreas Mueller, Hochschule Rapperswil
%
\chapter*{Einleitung\label{chapter:einleitung}}
\lhead{Einleitung}
\rhead{}
Die Entdeckung der Kugelgestalt der Erde bereits im Altertum hat auch
klar gemacht, dass die ebene Geometrie von Euklid nur bedingt für
die Lösung von geometrischen Problemen auf der Erdoberfläche geeignet
ist.
Die Gesetze der ebenen Geometrie sind nur anwendbar, solange 
die Abmessungen des Problems klein sind im Vergleich zum Erdradius.
Sobald die Abmessungen eines geometrischen Problems vergleichbar werden
mit dem Erdradius muss die Krümmung berücksichtigt werden.
Wir brauchen daher eine Geometrie, die auch für gekrümmte Flächen
funktioniert.

Einsteins allgemeine Relativitätstheorie hat gezeigt, dass sogar das
ganze Universum als gekrümmter Raum beschrieben werden muss, um die 
Wirkung der Gravitation korrekt zu verstehen.
Doch was soll das überhaupt heissen?
Es ist ja noch einigermassen intuitiv, sich die Erdoberfläche
als gekrümmte Fläche in einem dreidimensionalen Raum vorzustellen.
Doch wo ist das Universum eingebettet?
Dies ist jedoch die falsche Frage.
Die korrekte Frage ist: wie kann man die Krümmung der Eroberfläche
nur durch Messungen innerhalb der Erdoberfläche messen?
Oder allgemeiner: wie kann man die Krümmung des Universums nur durch
Messung im Universum festzustellen.
Krümmung ist daher nicht mehr eine Eigenschaft der Einbettung eines
Raumes in einen anderen, sondern sie äussert sich als eine Abweichung
der Längenmessungen von dem, was die euklidische Geometrie verspricht.

Es zeigt sich, dass der Begriff der Krümmung auch Anwendungen ausserhalb
der Geometrie hat.
Das Fermatsche Prinzip besagt zum Beispiel, dass sich Licht immer
den kürzesten Weg sucht.
In einem inhomogenen Medium ist die Lichtgeschwindigkeit nicht
an jedem Punkt gleich.
Für einen Lichtstrahl bedeutet das, dass die Längenmessung ortsabhängig wird.
Licht sucht sich also den kürzesten Weg in einem gekrümmten Raum.
Fast alle Naturgesetze lassen sich als Minimalprinzip formulieren.

Eine besondere Bedeutung hat die Beschreibung des Universums als gekrümmter
Raum natürlich auch deshalb, weil diese Theorie uns ermöglicht,
die Geschichte des Universums zu bestimmen.
Sie hilft uns daher auch, unseren Platz im Universum besser zu
verstehen.
Dazu wird ein Modell des Universums benötigt, welches die kleinräumige
Struktur wie Galaxien und Galaxienhaufen abstrahiert und nur noch
über die grossräumige Struktur eine Aussage macht.
In vielen naturwissenschaftlichen Anwendungen wird ähnlich vorgegangen.
Zum Beispiel abstrahiert die Zustandsgleichung eines idealen Gases
die einzelnen Moleküle des Gases und beschreiben die Gesamtheit des
Bewegungszustands durch wenige Grössen wie Druck, Temperatur und Volumen.
Ähnlich ermöglicht Reynolds-Averaging in der Strömungsdynamik die 
komplexe Bewegung der Wirbel durch ein einfaches zusätzliches Feld und
eine leichter zu lösende gemittelte Bewegungsgleichung zu berechnen.

Im zweiten Teil soll gezeigt werden, wie man durch zweckmässige
Vernachlässigung von Details und Mittelbildung in vielen Anwendungen
ein einfaches Modell für einen komplexen Sachverhalt bekommen kann.
Das Modell soll dabei immer noch in der Lage sein, wesentliche
Aspekte des ursprünglichen Problems beschreiben.
Dieses Prinzip soll dann auf das Universum angewendet werden.
Die Friedmann-Gleichungen erlauben uns, die Geschichte des
Universums und sein Alter zu bestimmen.

Eine Ungewissheit bleibt aber noch, nämlich die Frage, ob das
Universum als ganzes tatsächlich gekrümmt ist.
Die Antwort gibt es letzte mathematische Thema, nämlich die
sphärische harmonische Analyse.
Die Fouriertheorie erlaubt Funktion in einzelne Frequenzen zu
zerlegen.
Die sphärische harmonische Analyse ermöglich etwas ähnliches
für Funktionen auf einer Kugel. 
Sie bestimmt die typische Grösse von Features einer solchen
Funktion, ohne dass es auf die Position derselben auf der Kugeloberfläche
ankommt.

Damit lassen sich verschiedene Big Bang Modelle miteinander
vergleichen.
Es stellt sich heraus, dass das Universum im grossen keine Krümmung hat,
die Gesamtenergie des Universums ist daher $0$.
Dies bedeutet, dass die Entstehung eines Universums möglich ist,
ohne dass der Energieerhaltungssatz verletzt wird.




