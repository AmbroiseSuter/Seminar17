%
% k-gravitatzion.tex
%
% (c) 2017 Prof Dr Andreas Müller, Hochschule Rapperswil
%
\chapter{Gravitation%
\label{skript:kruemmusng:sectipn:gravitation}}
\lhead{Gravitation}
\rhead{}
Albert Einstein hat erkannt, dass die Wirkung der Gravitation 
durch die Krümmung des Raumes beschrieben werden muss.
\index{Einstein, Albert}
In den folgenden Abschnitten geben wir einen Ausblick darauf, wie
die moderne Physik unseren Raum als einen gekrümmten Raum beschreibt.
An einfachen Modellen soll gezeigt werden, wie man sich die Gravitationswirkung
als die Wirkung eines gekrümmten Raumes vorstellen kann, wie schwarze Löcher
beschrieben werden können, und wie alle diese Dinge tatsächlich gemessen
werden können.

\section{Äquivalenzprinzip}

\section{Newtonsches Gravitationsgesetz}
Als ersten Schritt in Richtung auf eine allgemeine Gravitationstheorie
zeigen wir, wie auch die Bahnen in einem Newtonschen Gravitationsfeld 
als Geodäten eines Zusammenhangs schreiben lassen.
Wer verwenden dazu aber nicht die Metrik und den daraus abgeleiteten 
Zusammenhang, dies hat erst die Einsteinsche Theorie geschafft.

Das Newtonsche Gravitationsgesetz beschreibt die Beschleunigung, die
auf ein Teilchen in einem Gravitationsfeld wirkt, welches von einer
Masse $m$ erzeugt wird.
Der Betrag der Kraft ist umgekehrt proportional zum Quadrat der
Entfernung.
Der Kraftvektor kann deshalb als
\[
\vec F = -\frac{GM}{r^2}\cdot\frac{\vec r}{r}
\]
geschrieben werden.
Die Bewegungsgleichung ist daher 
\begin{equation}
\ddot x^k = -\frac{GM}{r}\cdot\frac{x^k}{r},
\qquad\text{mit}\qquad r = \sqrt{(x^1)^2+(x^2)^2+(x^3)^2},
\label{skript:gravitation:bewegungsgleichung}
\end{equation}
es kommt nur die Beschleunigung und die Position vor.

In den Geodätengleichungen 
\[
\ddot x^\mu = -\Gamma^\mu_{\alpha\nu}\dot x^\alpha\dot x^\nu
\]
kommt dagegen auf der rechten Seite auch die Geschwindigkeit vor.
Es ist daher nicht unmittelbar klar, wie die Bewegungsgleichung als
Geodätengleichung geschrieben werden klar.

Wir beschreiben die Bewegung wieder in einem Vierdimensionalen Raum,
also als Kurve
\[
t\mapsto (t,x^1(t),x^2(t),x^3(t)),
\]
also mit $x^0(t)=t$.
Für eine solche Parametrisierung gilt $\dot x^0(t)=1$.
In der Geodätengleichung können wir daher die Terme mit mit Index $0$
separat behandelt werden:
\[
\ddot x^\mu
=
-\Gamma^\mu_{00}
-\Gamma^\mu_{k0}\dot x^k -\Gamma^\mu_{0l}\dot x^l
- \Gamma^\mu_{kl}\dot x^k\dot x^l
\]
Da in der Bewegungsgleichung~\eqref{skript:gravitation:bewegungsgleichung}
auf der rechten Seite keine Geschwindigkeiten vorkommen, darf nur der
erste Term stehen bleiben, also
\begin{equation}
\Gamma^k_{00} = \frac{GM}{r^2}\cdot \frac{x^k}{r},\qquad k=1,\dots,3.
\label{skript:gravitation:newtonzusammenhang}
\end{equation}
Da die erste Komponente der Geodäte $x^0(t)=t$ ist und seine zweite
Ableitung $\ddot x^0(t)=0$, muss auch $\Gamma^0_{\alpha\nu}=0$ sein.
Die einzigen nicht verschwindenden Zusammenhangskoeffizienten sind
\eqref{skript:gravitation:newtonzusammenhang}.

Man kann für den Zusammenhang~\eqref{skript:gravitation:newtonzusammenhang}
auch die Definition des Riemann-Krümmungstensors anwenden.
Die Rechnung zeigt aber, dass zwar einzelne Komponenten des Riemann-Tensors
nicht verschwinden, der Ricci-Tensor und damit auch der Einstein-Tensor
verschwinden dagegen identisch.
Diese Beschreibung der Gravitation führt also nicht auf das Modell
eines gekrümmten Raumes.

\section{Physikalische Bedeutung der Krümmung}

\section{Einstein-Tensor}

\section{Feldgleichungen}

