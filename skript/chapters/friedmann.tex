%
% friedmann.tex
%
% (c) 2017 Prof Dr Andreas Müller, Hochschule Rapperswil
%
\chapter{Friedmann-Gleichungen%
\label{skript:chapter:friedmann}}
\lhead{Friedmann-Gleichungen}
\rhead{}
Natürliche Systeme sind meistens so komplex, dass es praktisch nicht
möglich ist, jedes einzelne Detail physikalisch exakt wiederzugeben.
Es ist daher notwendig, vereinfachte Modelle zu verwenden, welche 
die Anzahl der zu berücksichtigenden Variablen reduzieren auf eine
Art, die immer noch gestattet, die gestellten Fragen mit ausreichender
Genauigkeit zu beantworten.

In diesem Kapitel betrachten wir zunächst ein paar Beispiele aus den
Naturwissenschaften, welche diesen Prozess der Modellbildung exemplarisch
vorstellen.
Anschliessend wird ein Modell für das Universum betrachtet, welches
das Universum derart vereinfacht, dass über die langfristige
Geschichte des Universums konkrete Aussagen zu machen gestattet.
Mit diesen sogenannten Friedmann-Gleichungen kann man sodann zum
Beispiel das Alter des Universums bestimmen.

\section{Das isotrope homogene Universum}
\rhead{Das isotrope homogene Universum}

\section{Friedmann-Gleichungen}
\rhead{Friedmann-Gleichungen}

\section{Zustandsgleichungen}
\rhead{Zustandsgleichungen}

\section{Geschichte des Universums}
\rhead{Geschichte des Universums}


