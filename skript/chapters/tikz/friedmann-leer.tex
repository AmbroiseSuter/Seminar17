%
% friedmann-leer.tex -- Entwicklung des Skalenfaktors in einem leeren Universum
%
% (c) 2017 Prof Dr Andreas Müller, Hochschule Rapperswil
%
\documentclass[tikz]{standalone}
\usepackage{times}
\usepackage{txfonts}
\usepackage{pgfplots}
\usetikzlibrary{arrows,intersections}
\begin{document}
\begin{tikzpicture}[thick,scale=3]
\coordinate (O) at (0,0);

\draw[->] (-1.05, 0   )--(3.3,  0  ) coordinate[label = {below:$t$}];
\draw[->] ( 1,   -0.05)--(1  ,  2.1) coordinate[label = {right:$a(t)$}];

\draw[red,line width=1.4] (-1,0)--(3,2);

\node at ( 1,0) [below right] {$t_0$};
\node at (-1,0) [below] {$\displaystyle (t-t_0) = -\frac1{H_0}$};

\end{tikzpicture}
\end{document}


