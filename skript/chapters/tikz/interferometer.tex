%
% interferometer.tex -- schematische Darstellung des Interferometers
%
% (c) 2017 Prof Dr Andreas Müller, Hochschule Rapperswil
%
\documentclass[tikz]{standalone}
\usepackage{times}
\usepackage{txfonts}
\usepackage{fp}
\usepackage{ifthen}
\usepackage[utf8]{inputenc}
\usetikzlibrary{arrows,intersections}
\usetikzlibrary{fixedpointarithmetic}
\begin{document}
\begin{tikzpicture}[>=latex, thick, scale=1.5]
\draw[red] (-1,0)--(5,0);
\draw[red] (0,-1)--(0,5);
\draw (-0.5,-0.5)--(0.5,0.5);
\node [label={right:halbdurchlässiger Spiegel}] at (0.5, 0.5) {};

\fill[gray!20] (5,-0.5)--(5.2,-0.5)--(5.2,0.5)--(5,0.5)--cycle;
\draw (5,-0.5)--(5,0.5);
\node[label={right:Spiegel}] at (5.2,0) {};

\fill[gray!20] (-0.5,5)--(-0.5,5.2)--(0.5,5.2)--(0.5,5)--cycle;
\draw (-0.5,5)--(0.5,5);
\node[label={above:Spiegel}] at (0,5.2) {};

\node[label={left:Quelle}] at (-1,0) {};
\draw (-1,-0.2)--(-1,0.2)--(-1.9,0.2)--(-1.9,-0.2)--cycle;

\node[label={[rotate=90] Detektor}] at (0.14,-1.6) {};
\draw (-0.2,-1)--(-0.2,-2)--(0.2,-2)--(0.2,-1)--cycle;

\end{tikzpicture}
\end{document}
