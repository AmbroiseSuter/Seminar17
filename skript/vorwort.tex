%
% vorwort.tex -- Vorwort zum Buch zum Seminar
%
% (c) 2015 Prof Dr Andreas Mueller, Hochschule Rapperswil
%
\chapter*{Vorwort}
\lhead{Vorwort}
\rhead{}
Dieses Buch entstand im Rahmen des Mathematischen Seminars
im Frühjahrssemester 2017 an der Hochschule für Technik Rapperswil.
Die Teilnehmer, Studierende der Abteilungen für Elektrotechnik,
Informatik und Bauingenieurwesen der
HSR, erarbeiteten nach einer Einführung in das Themengebiet jeweils
einzelne Aspekte des Gebietes in Form einer Seminararbeit, über
deren Resultate sie auch in einem Vortrag informierten. 

Im Frühjahr 2017 war das Thema des Seminars ``Mathematics for
the Universe''.
Es wurden drei mathematische Themengebiete besprochen, die für
das Verständnis des Universums unerlässlich sind, die aber auch
interessante Anwendungen in den Ingenieurwissenschaften haben.
Zur Sprache kam im ersten Thema das Konzept der Krümmung,
welches einerseits zentral für das Verständnis der Entwicklung
des Universums ist, aber auch Methoden für die Behanldung 
zum Beispiel der Geometrie auf gekrümmten Flächen bereitstellt.

Das zweite Thema ging von der für die Elektrotechnig wichtigen
Multipolzerlegung aus.
Daraus wurde dann die Analyse von Funktionen auf einer Kugeloberfläche,
die harmonische Analyse nach Kugelfunktionen entwickelt.
In der Kosmologie leitet man aus einer Analyse des kosmischen
Mikrowellenhintergrundes ab, dass das Universum flach ist.
In der Ingenieurpraxis gibt es Anwendungen dafür zum Beispiel
beim Registrierungsproblem \cite{skript:tabea}.

Das dritte Thema behandelt globale Modelle für das Universum.
Dabei wird eine Technik angewandt, die auch zum Beispiel in der
Strömungsmodellierung erfolgreich ist.
Beim Reynolds-Averaging mittelt die man zum Beispiel die Details
der turbulenten Strömung aus und ersetzt die Navier-Stokes-Gleichungen
durch einfacher zu lösende Gleichungen.
Dabei verliert man zwar auch einige Information, aber man gewinnt
ein Modell, das immer noch globale Aussagen über die Strömung
ermöglicht.
Im kosmologischen Zusammenhang kann man ein homogenes und isotropes
Universum mit der Friedmann-Gleichung modellieren.

Die Einführung bestand aus einigen Vorlesungsstunden, deren
Inhalt im ersten Teil dieses Skripts zusammengefasst ist.  

Im zweiten Teil dieses Skripts kommen dann die Teilnehmer selbst zu Wort.
Ihre Arbeiten wurden jeweils als einzelne
Kapitel mit meist nur typographischen Änderungen übernommen.
Diese weiterführenden Kapitel sind sehr verschiedenartig.
Eine Übersicht und Einführung befindet sich in der Einleitung
zum zweiten Teil auf Seite~\pageref{skript:uebersicht}.

In einigen Arbeiten wurde auch Code zur Demonstration der 
besprochenen Methoden und Resultate geschrieben, soweit
möglich und sinnvoll wurde dieser Code im Github-Repository
dieses Kurses\footnote{\url{https://github.com/AndreasFMueller/Seminar17.git}}
abgelegt, in anderen Fällen verweisen die Artikel selbst auf
das zugehörige Code-Repository.

Im genannten Repository findet sich auch der Source-Code dieses
Skriptes, es wird hier unter einer Creative Commons Lizenz
zur Verfügung gestellt.

